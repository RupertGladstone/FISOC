
\documentclass[12pt]{article}
%\documentclass[12pt,twocolumn]{article}

\usepackage{listings}
\usepackage{color}
\usepackage{hyperref}

\definecolor{dkgreen}{rgb}{0,0.6,0}
\definecolor{gray}{rgb}{0.5,0.5,0.5}
\definecolor{mauve}{rgb}{0.58,0,0.82}

\lstset{%frame=tb,
  language=bash,
  aboveskip=3mm,
  belowskip=3mm,
  showstringspaces=false,
  columns=flexible,
  basicstyle={\small\ttfamily},
  numbers=none,
  numberstyle=\tiny\color{gray},
  keywordstyle=\color{blue},
  commentstyle=\color{dkgreen},
  stringstyle=\color{mauve},
  breaklines=true,
  breakatwhitespace=true,
  tabsize=3
}

\begin{document}

\title{Framework for Ice Sheet - Ocean Coupled modelling (FISOC) Manual}
\author{Rupert Gladstone (RupertGladstone1972@gmail.com) \and Ben Galton-Fenzi}
\date{Version 0.0, April 2015}
\maketitle


\section{Introduction}

The ``Framework for Ice Sheet - Ocean model Coupling" (FISOC) has been written to enable Ice Sheet Models 
(ISMs) and Ocean Models (OMs) to be run as a single executable to address the co-evolution of ice and ocean 
properties.  it is primarily designed to deal with the situation of a floating ice shelf at the interface 
between land ice and open ocean.

In this context an ISM simulates (part of) a marine ice sheet, including both grounded and floating parts, 
representing the dynamic evolution over time of the ice sheet.

An OM simulates the sub-shelf cavity circulation under the floating part of the ice sheet, and optionally also 
a wider ocean domain.

FISOC is a set of code modules and driver built using the Earth System Modelling Framework (ESMF, 
\url{https://www.earthsystemcog.org/projects/esmf/}). 
Some knowledge of the ESMF is essentially in order to fully understand the FISOC code.  It should 
be possible to run FISOC without knowledge of ESMF.

A full description of the physical processes FISOC attempts to simulate are provided in (***
ref GMD paper, not yet written).  
This manual describes how to use FISOC with an ISM and OM for which it has been 
developed (Section~\ref{sec:FISOC_SUG}) and also how to build additional ISM or OM components 
into the FISOC framework (Section~\ref{sec:FISOC_SDG}).




\section{Using FISOC with established components}
\label{sec:FISOC_SUG}

\subsection{Installing FISOC}

FISOC can be obtained by emailing a request to the author.  It is maintained and developed in a 
private GitHub repository.  We intend to make this public in 2016.

FISOC has a simple build process.  The Makefile in the top level FISOC directory contains the 
hard coded dependencies needed to build FISOC code.  The Makefile refers to certain 
environment variables to determine paths and component choices (Section~\ref{sec:EnvVars}). 

Having installed the pre-requisites (Section~\ref{sec:PreReq}), simply run at the command line, 
in the top level FISOC directory:
\begin{lstlisting}
make install
\end{lstlisting}


\subsubsection{FISOC Environment Variables}
\label{sec:EnvVars}

Several environment variables may be used in the build process. 
Some of these are mandatory. 
Variables listed as optional here may be mandatory for configuratons other than 
``dummy''.
These environment variables are not used at run time, only during 
the compilation/installation of FISOC.

\begin{flushleft}
\textbf{FISOC\_INSTALL\_DIR}                       \\ 
Optional.                                          \\
Determines where FISOC\_caller will be installed. Defaults to \$HOME/bin. 
The user should also ensure this location is in their \$PATH. \\
\vspace{6pt}
\textbf{FISOC\_OM}                                 \\ 
Required. Possible values ``dummy'', ``ROMS''.     \\
Determines which OM component will be used.        \\
\vspace{6pt}
\textbf{FISOC\_OM\_INCLUDE}                       \\ 
Optional.                                          \\
Specifies the path where the OM header files are located.\\
\vspace{6pt}
\textbf{FISOC\_OM\_LIBPATH}                       \\
Optional.                                          \\
Specifies the path where the OM library files are located.\\
\vspace{6pt}
\textbf{FISOC\_OM\_LIBS}                          \\
Optional.                                          \\
Specifies the linker directives needed to link the OM library to FISOC. \\
\vspace{6pt}
\textbf{FISOC\_ISM}                                \\
Required. Possible values ``dummy'', ``Elmer''.    \\
Determines which ISM component will be used.       \\
\vspace{6pt}
\textbf{FISOC\_ISM\_INCLUDE}                       \\ 
Optional.                                          \\
Specifies the path where the ISM header files are located.\\
\vspace{6pt}
\textbf{FISOC\_ISM\_LIBPATH}                       \\
Optional.                                          \\
Specifies the path where the ISM library files are located.\\
\vspace{6pt}
\textbf{FISOC\_ISM\_LIBS}                          \\
Optional.                                          \\
Specifies the linker directives needed to link the ISM library to FISOC. \\
\end{flushleft}



\subsubsection{Pre-requisites}
\label{sec:PreReq}

FISOC has been tested with GNU compilers on Linux Mint using OpenMPI. 

A Message Passing Interface (MPI) implementation, such as 
OpenMPI (\url{http://www.open-mpi.org/}), is required. 

The Network Common Data Form (NetCDF,  
\url{http://www.unidata.ucar.edu/software/netcdf/}) 
Fortran interface must be available 

ESMF must be available (\url{https://www.earthsystemcog.org/projects/esmf/}).  
ESMF should have been built with NetCDF and MPI 
(see also notes in Appendix~\ref{app:A}).

%For example, environment variables like these may be used for the ESMF build
%\begin{lstlisting}
%export ESMF_NETCDF="split"
%export ESMF_NETCDF_INCLUDE="/usr/local/include/"
%export ESMF_COMM="openmpi"
%\end{lstlisting}

Viable ISM and OM components must be available for any physically meaningful simulations
(the build may be tested using ``dummy'' components).  
See Sections~\ref{sec:Elmer} and \ref{sec:ROMS}.


\subsection{Running FISOC}

The FISOC executable is called FISOC\_caller, and should be located in your path after installation. 
The installation is specific to the choice of component (you need to re-compile if you switch , for 
example, from one ISM to another).  
Beyond choice of component, all other run choices are made in the FISOC\_config.rc file
(Section~\ref{sec:config}), 
or through component specific initialisation.

In the first instance a dummy coupler can be run be setting both environment variables FISOC\_ISM and 
FISOC\_OM to ``dummy'' at compile time.  This can help to test the compilation, and was used during development, 
but performs no meaningful science.  

In verbose mode (Section~\ref{sec:config}) some run time information may be printed to the screen.  
Independently of this, log files are always written, 
with default filenames of ``PET\#.FISOC.Log'', where ``\#'' is a number from 0 upwards indicating the 
``Persistent Execution Thread'' (PET). 
These logs containing run time messages and errors can be helpful with trobleshooting.
Note that FISOC appends to the logs rather than over-writing at run time, so you may wish to delete old logs 
periodically. 

Aside: PET is an ESMF abstraction designed to be general over differing parallel implementations. 
In FISOC, there is always a 1:1 relationship between PETs and MPI processes 
(Section~\ref{sec:FISOC_SDG}). 


\subsubsection{FISOC configuration}
\label{sec:config}

The FISOC config file is named FISOC\_config.rc.  This is a resource file, as described by the 
ESMF documentation.  It supports different types and also lists. 
In principle, llists of mixed types are supported, though FISOC does not utilise this utility.

FISOC\_config.rc contains code specific to the coupling.  Components may also use their 
independent means for initialisation.

Some of the FISOC config entries are required.
This section describes all the valid standard entries in a FISOC config file as follows:

\begin{flushleft}
\textbf{label:}             [TYPE]   [Required?]                          \\
Description                                                               \\
\vspace{6pt}
\vspace{6pt}
\textbf{ISM\_meshFile:}      [STRING] [optional]                          \\
The name of a netcdf file containing the ISM mesh in ESMF format.         \\
\vspace{6pt}
\textbf{ISM\_configFile:}    [STRING] [optional]                          \\
The name of the ISM-specific config file.                                 \\
\vspace{6pt}
\textbf{FISOC\_ISM\_ReqVars:} [STRING] [required]                         \\
List of variable names required to be provided by the ISM.                \\
\vspace{6pt}
\textbf{ISM\_ReqVars:}       [STRING] [optional]                          \\
Corresponding exactly to the above, but model-specific in case of different naming conventions. \\
\vspace{6pt}
\textbf{FISOC\_ISM\_DerVars:} [STRING] [required]                         \\
List of variables derived by FISOC from the ISM vars.                     \\ 
To be calculated from ISM vars by hard coded routines in FISOC\_ISM.      \\
\vspace{6pt}
\textbf{OM\_configFile:}     [STRING] [optional]                          \\
\vspace{6pt}
\textbf{FISOC\_OM\_ReqVars:}  [STRING] [required]                         \\
List of variable names required to be provided by the OM.                 \\
\vspace{6pt}
\textbf{OM\_ReqVars:}        [STRING] [optional]                          \\
Corresponding exactly to the above, but model-specific in case of different naming conventions. \\
\vspace{6pt}
\textbf{FISOC\_OM\_DerVars:}  [STRING] [required]                         \\
List of variables derived by FISOC from the OM vars.  
To be calculated from ISM vars by hard coded routines in FISOC\_OM.       \\
\vspace{6pt}
\textbf{OM\_outputInterval:} [INTEGER][optional]                          \\
FISOC collects OM output once every OM\_outputInterval OM timesteps. 
Defaults to 1.  dt\_ratio/OM\_outputInterval must be integer.             \\
\vspace{6pt}
\textbf{OM\_dt\_sec:}         [INTEGER][required]                         \\
OM timestep length in seconds.                                            \\
\vspace{6pt}
\textbf{dt\_ratio:}          [INTEGER][required]                          \\
ISM/OM timestep ratio.                                                    \\
\vspace{6pt}
\textbf{start\_year:}        [INTEGER][required]                          \\
Start year and month define the start time of the coupled simulation.     \\
\vspace{6pt}
\textbf{start\_month:}       [INTEGER][required]                          \\
\vspace{6pt}
\textbf{end\_year:}          [INTEGER][required]                          \\
End year and month define the finish time of the coupled simulation.      \\
\vspace{6pt}
\textbf{end\_month:}         [INTEGER][required]                          \\
\vspace{6pt}
\textbf{verbose\_coupling:}  [LOGICAL][required]                          \\
Print FISOC comments to screen (a log will be written independently of the settings of this value).\\
\end{flushleft}

\subsubsection{Timestepping}
Asynchronous timestepping.
***fill in this section we've implemented both tight coupling (ice and ocean both running 
on the same timescale) and loose coupling (for longer time scales where the ocean is run 
to steady state then the ice sheet continues until significant change has occurred in the cavity).

\subsection{Elmer/Ice}
\label{sec:Elmer}
***ref or link Elmer/Ice

FISOC needs to know where to find the relevant Elmer/Ice libraries.  
This can be done at FISOC compile time through the 
\$FISOC\_ISM
environment variables.  

For dynamic linked libraries, shared object files may be needed at run time.  
This can be ensured through use of 
the \$LD\_LIBRARY\_PATH environment variable. 

For example (it is assumed \$ELMER\_HOME was set during Elmer installation):
\begin{lstlisting}
export FISOC_ISM="Elmer"
export FISOC_ISM_INCLUDE="$ELMER_HOME/share/elmersolver/include"
export FISOC_ISM_LIBPATH="$ELMER_HOME/lib/"
export FISOC_ISM_LIBS="-lelmersolver"
export LD_LIBRARY_PATH="$FISOC_ISM_LIBPATH/:$LD_LIBRARY_PATH"
\end{lstlisting}


\subsection{ROMS}
\label{sec:ROMS}

FISOC has been developed and tested with an ice shelf enabled version of ROMS. 
This is branched from the Rutgers ROMS repository and is publicly available 
through github: \\
https://github.com/bkgf/romsIceShelf

When compiling ROMS for use with FISOC, It is essential to activate the option 
to compile the ROMS shared library, which 
is done by setting the environment variable \$ROMS\_MAKE\_SHAREDLIB to any value. 
The shared library will be installed in the location given by the 
 \$LIBDIR environment variable. 
By default ROMS will install the module files in the directory given by 
 \$SCRATCH\_DIR.  

Similarly to the ISM options, FISOC uses the variables \$FISOC\_OM\_LIBPATH and
 \$FISOC\_OM\_LIBS to locate the shared library.
FISOC uses the  \$FISOC\_OM\_INCLUDE variable to locate the module files, 
which are needed during compilation.

FISOC needs to access the shared library at run time.  One way of ensuring this 
is to set the \$ LD\_LIBRARY\_PATH variable, e.g.:
\begin{lstlisting}
export LD_LIBRARY_PATH="$FISOC_OM_LIBPATH/:$LD_LIBRARY_PATH"
\end{lstlisting}

\section{Adapting new components to run with  FISOC}
\label{sec:FISOC_SDG}

Any new OM or ISM component to be used with FISOC must first be ESMF compliant.  This basically 
means that it should have an initialise, run and finalise routine. 

FISOC is designed such that the only code developments for new components should be made to the 
model-specific wrappers: FISOC\_OM\_Wrapper\_XXX.f90 and FISOC\_ISM\_Wrapper\_XXX.f90, where 
XXX should be replaced by the component's name.

If it is found that changes to other aspects of the FISOC code are required, this should be 
implemented in collaboration with the FISOC lead developers.

If you wish to submit your new model-specific wrapper to the FISOC repository, please contact 
the FISOC lead developers.


\subsection{Coding practices}

New code should ideally be in Fortran 90 modules.  
All moules should contain the ``implicit none'' statement at the top (immediately after any 
``USE'' statements).  This property will be inherited by all procedures in the module.

Modules should ideally have the private attribute, with only required procedures being 
made public. 

New model-specific wrapper modules should ensure that the initialise, run and finalise 
subroutines are public. 
When writing new wrapper modules, the existing wrapper modules may be used as a template, 
providing examples of the required interfaces.


\subsection{Configuration options}

The configuration file must be called ``FISOC_config.rc'' and is compatible with ESMF 
config methods.  
An ESMF\_config object is automatically created from this file.

There may be some parameters that are derived from config parameters.  
The FISOC\_utils module provides subroutines under the 
FISOC\_ConfigDerivedAttribute interface 
for obtaining derived config parameters.
These subroutines can be viewed informally as additional methods to complement the ESMF config access methods. 

An advantage of the way the config object is used is that new config arguments can be added and 
used in the coupling without requiring developments to any code other than the new wrapper. 
The config object can be passed to the wrapper and accessed directly. 


\subsection{ISM wrapper}

\subsection{OM wrapper}

\section{Future developments}

\appendix

\section{Pre-requisite installation notes}
\label{app:A}
The following commands worked to install NetCDF and ESMF on a Linux Mint system in Feb 2015.
Some pre-requisites for netcdf were also installed.

\begin{lstlisting}
cd /somewhere/to/download/and/compile/source/code

sudo apt-get install m4

wget ftp://ftp.unidata.ucar.edu/pub/netcdf/netcdf-4/
                                                        zlib-1.2.8.tar.gz
tar -xzf zlib-1.2.8.tar.gz 
cd zlib-1.2.8
 ./configure --prefix=/usr/local/
 make check
 sudo -E make install
cd ..

wget ftp://ftp.unidata.ucar.edu/pub/netcdf/netcdf-4/
                                                       hdf5-1.8.13.tar.gz	
tar -xzf hdf5-1.8.13.tar.gz 
cd hdf5-1.8.13
 # Note the O0 flag in the next line.  The default is O3, which is
 # strong optimisation.  This can result in failed checks on some
 #  systems.
 CC=mpicc FC=mpif90 ./configure  --prefix=/usr/local/ --with-zlib=/usr/local  --enable-fortran --enable-parallel

 CFLAGS=-O0 ./configure --prefix=/usr/local/ --with-zlib=/usr/local  --enable-fortran 
 make check
 sudo -E make install
cd ..

wget ftp://ftp.unidata.ucar.edu/pub/netcdf/netcdf-4.3.3.tar.gz
tar -xzf netcdf-4.3.3.tar.gz 
cd netcdf-4.3.3/
 CPPFLAGS="-I/usr/local/include -I/usr/local/hdf5/include" LDFLAGS="-L/usr/local/hdf5/lib -L/usr/local/lib" ./configure --prefix=/usr/local 
 CPPFLAGS=-I/usr/local/include LDFLAGS=-L/usr/local/lib ./configure --prefix=/usr/local 
 make check
 sudo -E make install
cd ..

wget ftp://ftp.unidata.ucar.edu/pub/netcdf/
                                              netcdf-fortran-4.4.2.tar.gz
tar -xzf netcdf-fortran-4.4.2.tar.gz 
cd netcdf-fortran-4.4.2
 LD_LIBRARY_PATH=/usr/local/lib/ CPPFLAGS="-I/usr/local/include  -I/usr/local/hdf5/include -DUSE_NETCDF4" LDFLAGS="-L/usr/local/lib -L/usr/local/hdf5/lib" ./configure  --prefix=/usr/local
 LD_LIBRARY_PATH=/usr/local/lib/ CPPFLAGS="-I/usr/local/include -DUSE_NETCDF4" LDFLAGS="-L/usr/local/lib" ./configure  --prefix=/usr/local
 make check
 sudo -E make install
cd ..

export ESMF_DIR="/top/level/directory/for/esmf/"
export ESMF_NETCDF="split"
export ESMF_NETCDF_INCLUDE="/usr/local/include"
export ESMF_NETCDF_LIBPATH="/usr/local/lib"
export ESMF_COMM="openmpi"
export ESMFMKFILE="$ESMF_DIR/DEFAULTINSTALLDIR/lib/libO/
                             Linux.gfortran.64.openmpi.default/esmf.mk"
                                                                                                              
wget downloads.sourceforge.net/project/esmf/ESMF_6_3_0r/
                                 ESMF_6_3_0rp1/esmf_6_3_0rp1_src.tar.gz
                                                                                                              
tar -xf esmf_6_3_0rp1_src.tar.gz
cd esmf 
 make check
 sudo -E make install
cd ..

\end{lstlisting}

\end{document}
