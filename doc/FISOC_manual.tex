
\documentclass[11pt]{article}
%\documentclass[12pt,twocolumn]{article}

\usepackage{listings}
\usepackage{color}
\usepackage{hyperref}
\usepackage[margin=0.6in]{geometry}
\usepackage{graphicx}
\usepackage{wrapfig}

\definecolor{dkgreen}{rgb}{0,0.6,0}
\definecolor{gray}{rgb}{0.5,0.5,0.5}
\definecolor{mauve}{rgb}{0.58,0,0.82}

\lstset{%frame=tb,
  language=bash,
  aboveskip=3mm,
  belowskip=3mm,
  showstringspaces=false,
  columns=flexible,
  basicstyle={\small\ttfamily},
  numbers=none,
  numberstyle=\tiny\color{gray},
  keywordstyle=\color{blue},
  commentstyle=\color{dkgreen},
  stringstyle=\color{mauve},
  breaklines=true,
  tabsize=3, 
  postbreak=\raisebox{0ex}[0ex][0ex]{\ensuremath{\color{red}\hookrightarrow\space}}
}

\begin{document}

\title{Framework for Ice Sheet - Ocean Coupled modelling (FISOC) Manual, V1.1}
\author{Rupert Gladstone and other FISOC contributors  (fisoc\_dev@googlegroups.com)}
%\and Lenneke Jong \and Ben Galton-Fenzi \and David Gwyther}
%\date{Version 0.2, April 2016}

\maketitle

\newpage 
\tableofcontents
\newpage 

\section{Introduction}

The ``Framework for Ice Sheet - Ocean model Coupling" (FISOC) has been written to enable Ice Sheet Models 
(ISMs) and Ocean Models (OMs) to be run as a single executable to address the co-evolution of ice and ocean 
properties.  
It is primarily designed to handle exchange of variables between ice and ocean at the underside of a floating ice shelf.
At time of writing (May 2020) exchange of variables at an ice cliff is 
not supported, though this is a planned development.

In this context an ISM simulates (part of) a marine ice sheet, including both grounded and floating parts, 
representing the dynamic evolution over time of the ice sheet.

An OM simulates the sub-shelf cavity circulation under the floating part of the ice sheet, and optionally also 
a wider ocean domain.

FISOC comprises a set of code modules and driver built using the Earth System Modelling Framework (ESMF, 
\url{https://www.earthsystemcog.org/projects/esmf/}), and written in Fortran 90. 
Some knowledge of the ESMF is essential in order to fully understand the FISOC code.  It should 
be possible to run FISOC as an end-user without knowledge of ESMF.
FISOC is intended to be flexible: the additional development effort required to 
couple a different ocean or ice component into the framework is minimal, so 
long as the new component is ESMF-compatible. 

FISOC initially couples the ISM Elmer/Ice to the OM ``Regional Ocean Modelling System'' (ROMS).
A full description of the physical processes represented in FISOC in its 
initial configuration is given in a model description paper (to be submitted
to Geoscientific Model Development, GMD, during 2020).

This manual describes how to use FISOC with an ISM and OM for which it has been 
developed (Section~\ref{sec:FISOC_SUG}) and also how to integrate 
additional ISM or OM components 
into the FISOC framework (Section~\ref{sec:FISOC_SDG}).

A number of web links are given at various points in this document.  
The links were correct at the time they were added. 
Please contact the developers if any of these are found to be out of date. 

\subsection{FISOC community and contact}

Three email groups exist within the FISOC project.

\textbf{fisoc@googlegroups.com} is for news and updates. 
It has low traffic. 
Anyone may apply to join this group. 

\textbf{fisoc\_tech@googlegroups.com} is for active users and developers to
share technical issues and troubleshooting.
It is higher traffic.
Anyone may apply to join this group. 

\textbf{fisoc\_dev@googlegroups.com} is the contact address to get in touch
with the FISOC developers.
Membership is by invitation only, but anyone may post to, and read, this group's contents.




\section{Installing FISOC with established components}
\label{sec:FISOC_install}

FISOC can be obtained directly from a public github repository: \\
\url{https://github.com/RupertGladstone/FISOC}

The most active branch is devel, and master is updated from this infrequently. 

FISOC has a simple build process.  The Makefile in the top level FISOC directory contains the 
hard coded dependencies needed to build FISOC code.
The Makefile refers to certain 
environment variables to determine paths and component choices (Section~\ref{sec:EnvVars}). 
An example script to build and install FISOC, ``buildFISOCexample.sh'', is available in the top 
level FISOC directory.
The script simply sets appropriate environment variables then calls the Make process. 
The pre-requisites (Section~\ref{sec:PreReq}) must already be installed. 

%Having installed the pre-requisites (Section~\ref{sec:PreReq}), simply run at the command line, 
%in the top level FISOC directory:
%\begin{lstlisting}
%make install
%\end{lstlisting}

FISOC has been built and used with GNU and with Intel compilers. 
This has been carried out on different flavours of Linux.
Automated testing to validate an install is planned for the future. 
%The Message Passing Interface (MPI) implementation is OpenMPI. 
%In principal it should also work with MPICH, but this has not been tested at time of writing (5/12/2017). 



\subsection{Switching between components}

Many aspects of FISOC are runtime configurable (Section~\ref{sec:FISOC_SUG}), 
but the choice of which component to use is a compile time option, implemented 
by choice of model-specific wrapper. 
This is determined by environment variables FISOC\_OM and FISOC\_ISM 
(Section~\ref{sec:EnvVars}). 

In order to test the build process the dummy wrappers may be used. 
Libraries are not needed for dummy wrappers, and the corresponding 
environment variables can be set to any value.
See  ``buildFISOCexample.sh'' for examples. 

In addition to the dummy wrappers, undocumented ISM components 
 include: 
``FOrcing OffLine'' (FOOL), which acts as a wrapper for netcdf files 
containing time evolving geometry forcing we want to use to force the 
OM; 
``Frank's Ice Shelf model'' (FISh), a very simple flowline SSA 
ice shelf model from Frank Pattyn, used for simple testing during 
development.
There is also an undocumented OM component, the Finite Volume Community Ocean Model (FVCOM), which is newly coupled through
FISOC and will be documented soon (intended during 2020). 



\subsection{FISOC Environment Variables}
\label{sec:EnvVars}

A number of  environment variables, summarised in Table~\ref{tab:envvars},
may be used in the build process.
Some of these are mandatory. 
Variables listed as optional in Table~\ref{tab:envvars}
may be mandatory for some configurations 
other than ``dummy''.
These environment variables are not used at run time, only during 
the compilation/installation of FISOC.

\begin{table}
  \begin{center}
    \begin{tabular}{ l|l|l|l }
      \textbf{Environment  variable}    &                    & \textbf{Description}                    & \textbf{Default} \\
      \hline
      \textbf{ESMFMKFILE}          & Required           & Tells FISOC where to find ESMF          & None   \\
      \textbf{FISOC\_INSTALL\_DIR} & Optional           & Determines where FISOC executable       & \$HOME/bin \\ 
                                   &                    & will be installed (should be in \$PATH) & \\
      \textbf{FISOC\_EXE}          & Optional           & Name of the FISOC executable            & FISOC\_caller. \\
      \textbf{CPPFLAGS}            & Optional           & Preprocessor directives                 & None \\
      \hline
      \textbf{FISOC\_ISM}          & Required           & Determines ISM component                  & None \\
                                   &                    & (``dummy'', ``Elmer'' or ``FOOL'')        &      \\
      \textbf{FISOC\_ISM\_INCLUDE} & Optional           & Location of ISM header files              & None    \\
      \textbf{FISOC\_ISM\_LIBPATH} & Optional           & Location of ISM library files             & None    \\
      \textbf{FISOC\_ISM\_LIBS}    & Optional           & Linker directives for ISM library         & None    \\
      \textbf{FISOC\_ISM\_GEOM}    & Optional           & ESMF geometry object for ISM              & FISOC\_ISM\_MESH \\
                                   &                    & (FISOC\_ISM\_GRID or                      & \\
                                   &                    & FISOC\_ISM\_MESH)                         & \\
      \hline
      \textbf{FISOC\_OM}           & Required           & Determines OM component                   & None    \\
                                   &                    & (``dummy'', ``ROMS'' or ``FVCOM'')        &         \\
      \textbf{FISOC\_OM\_INCLUDE}  & Optional           & Location of OM header files               & None    \\
      \textbf{FISOC\_OM\_LIBPATH}  & Optional           & Location of OM library files              & None    \\
      \textbf{FISOC\_OM\_LIBS}     & Optional           & Linker directives for OM library          & None    \\
      \textbf{FISOC\_OM\_GEOM}     & Optional           & ESMF geometry object for OM               & FISOC\_OM\_GRID \\
                                   &                    & (FISOC\_OM\_GRID or                       & \\
                                   &                    & FISOC\_OM\_MESH)                          & \\
    \end{tabular}
  \end{center}
  \caption{
    Environment variables that are used in the FISOC build process.
    Note that \textbf{CPPFLAGS} should include \textbf{FISOC\_MPI} for parallel runs. 
  }
  \label{tab:envvars}
\end{table}

%\vspace{6pt}
% now set through cpp flags
%\textbf{FISOC\_MPI}                                \\ 
%Optional. Possible value ``yes''.                  \\
%Sets preprocessor flag to tell FISOC whether this is a parallel compilation.
%Defaults to serial.                                \\

See also Section~\ref{sec:PreReqROMS} for ROMS-specific preprocessor directives.

%\subsubsection{Grids and meshes}
%OM\_gridType and ISM\_gridType refer to the type of ESMF object to be used for holding information about 
%the model grid.  Typically models utilising unstructured meshes (e.g. Elmer/Ice) would use an 
%ESMF\_mesh object for holding mesh information in FISOC and models utilising structured grids 
%(e.g. ROMS) would use an ESMF\_grid object for holding grid information in FISOC.


\subsection{Pre-requisites}
\label{sec:PreReq}

A Message Passing Interface (MPI) implementation, such as 
OpenMPI is required.\\
 \url{http://www.open-mpi.org/}

The Network Common Data Form (NetCDF) Fortran interface must be available. 
More specifically, NetCDF4 in parallel should be used (this is not the same as PnetCDF). \\
\url{http://www.unidata.ucar.edu/software/netcdf/}

ESMF must be available. ESMF should have been built with NetCDF and MPI 
(see also notes in Appendix~\ref{app:A}). \\
\url{https://www.earthsystemcog.org/projects/esmf/}.

%For example, environment variables like these may be used for the ESMF build
%\begin{lstlisting}
%export ESMF_NETCDF="split"
%export ESMF_NETCDF_INCLUDE="/usr/local/include/"
%export ESMF_COMM="openmpi"
%\end{lstlisting}

Viable ISM and OM components must be available for any physically meaningful simulations
(the build may be tested using ``dummy'' components).  
See Sections~\ref{sec:PreReqElmer} and \ref{sec:PreReqROMS}.


\subsubsection{Elmer/Ice}
\label{sec:PreReqElmer}

\textbf{Compiling Elmer/Ice for FISOC} \\
At time of writing (May 2020) FISOC requires a non-standard Elmer/Ice 
code branch.
The Elmer repository is here:
\url{https://github.com/ElmerCSC/elmerfem/}.
The standard branch used by glaciologists is the elmerice branch. 
The branch needed for FISOC is the elmerice\_FISOC branch. 
It is intended to merge the changes during 2020.
You can checkout the relevant branch with a command like this:
\begin{lstlisting}
git clone git://www.github.com/ElmerCSC/elmerfem -b elmerice_FISOC MyLocalName
\end{lstlisting}

There is nothing FISOC-specific about the Elmer/Ice build process.  
So long as you have the FISOC-compatible code branch, you can build Elmer/Ice 
just the same as normal. 
The Elmer/Ice wiki has information on a standard build:
\url{http://elmerice.elmerfem.org/wiki/doku.php?id=compilation:compilationcmake}

\vspace{10pt}

\textbf{Compiling FISOC with Elmer/Ice}\\
When compiling FISOC with Elmer/Ice, FISOC needs to know where to 
find the relevant Elmer/Ice libraries.  
This can be done at FISOC compile time through the 
\$FISOC\_ISM
environment variables.  For example:

\begin{lstlisting}
export FISOC_ISM="Elmer"
export FISOC_ISM_INCLUDE="$ELMER_HOME/share/elmersolver/include"
export FISOC_ISM_LIBPATH="$ELMER_HOME/lib/"
export FISOC_ISM_LIBS="-lelmersolver"
\end{lstlisting}





\subsubsection{ROMS}
\label{sec:PreReqROMS}

\textbf{Compiling ROMS for FISOC} \\
FISOC has been developed and tested with an ice shelf enabled version of ROMS. 
This is branched from the Rutgers ROMS repository.  Information about the Rutgers ROMS 
can be found at \url{https://www.myroms.org/}.

Development of the ice shelf enabled version  is currently ongoing 
in a private repository 
(please contact the developers if you need access to this).
%A public version of the ice shelf enabled version is also available 
%through github: 
%\url{https://github.com/bkgf/romsIceShelf}.  
%Please use the ``FISOC\_friendly'' branch.
%At time of writing (5/12/2017) the public version is out of date and 
%the private version should be used. 
%It is intended to push changes to the public version at the time of 
%submission of the FISOC GMD paper, during 2018.

Example build scripts can be found in the ROMS/Bin subdirectory of a git clone 
from the repository mentioned above.
It is assumed that users are familiar with a standard ROMS build process. 


When compiling ROMS for use with FISOC, the following additional environment
variables are needed:
\begin{lstlisting}
export MAKE_SHAREDLIB="Yes"
export LIBDIR="/usr/local/lib"
export MY_CPP_FLAGS=" -DFISOC"
\end{lstlisting}

It is essential to activate the option to compile the ROMS shared library, which 
is done by setting the environment variable MAKE\_SHAREDLIB to any value. 
The -fPIC flag is essential, and this will in general be activated 
by MAKE\_SHAREDLIB in a makefile supplement included by a line like 
this in the makefile:
\begin{lstlisting}
include $(COMPILERS)/$(OS)-$(strip $(FORT)).mk
\end{lstlisting}
This should simply work, but may require minor modifications on new 
OS/compiler combinations.

The shared library will be installed in the location given by the 
 LIBDIR environment variable. 

The -DFISOC flag activates FISOC-specific code segments.  
This includes telling ROMS to use a 
specific hard coded unit rather than outputting to screen.  This relies on the 
same unit being hard coded in the FISOC ROMS wrapper, and results in the ROMS 
messages being sent to file instead of printed to screen.

Some aspects of the ROMS setup for specific simulations are determined at 
compile time rather than run time. 
These are determined through cpp directives set in an application-specific 
header file in ROMS/Include in the ROMS clone. 
ROMS must be compiled with an ``application'' (this is set through an 
environment variable, see example ROMS build script) that has been setup for 
use with FISOC. 
This should give a good indication of which applications are FISOC-compatible:
\begin{lstlisting}
grep ``ifdef FISOC'' ROMS/Include/*h
\end{lstlisting}

The cpp options are described in ROMS/Include/cppdefs.h. 
More detail of the FISOC specific options is given below, but to compile ROMS 
for use with 
existing FISOC examples, knowledge of these is not required as the header 
files will set the required cpp directives and the header file is determined by 
choice of ROMS application.

\begin{flushleft}
To tell ROMS to expect FISOC to provide a cavity change rate:\\
FISOC\_DDDT\\
To tell ROMS to expect FISOC to provide an upper ice surface change rate
(required for grounding line migration 
and is used to update the wet/dry masks based on floatation):\\
FISOC\_DSDT\\
To tell ROMS to expect FISOC to directly overwrite the iceshelf draft 
(this and FISOC\_DDDT are mutually exclusive):\\
FISOC\_DRAFT\\
To tell ROMS to expect to receive a vertical temperature gradient 
at the ice base from FISOC: \\
FISOC\_DTDZ\\
To tell ROMS to calculate the averages of the variables provided to FISOC at 
the end of each ROMS run call:\\
ROMS\_AVERAGES\\
If ice shelf geometry evolution is required in ROMS this options must be 
defined:\\
ICESHELF\_MORPH\\
If grounding line movement is required in ROMS the following cpp options 
must be defined:\\
LIMIT\_BSTRESS\\
LIMIT\_ICESTRESS\\
WET\_DRY
\end{flushleft}


By default ROMS will install the module files in the directory given by 
 SCRATCH\_DIR.  

\vspace{10pt}
\textbf{Compiling FISOC with ROMS}\\
When compiling FISOC with ROMS, FISOC needs to know where to 
find the relevant ROMS libraries.  
This can be done at FISOC compile time through the 
\$FISOC\_OM
environment variables.  For example:

\begin{lstlisting}
export MY_ROMS_DIR="/home/elmeruser/Source/ROMSIceShelf_devel"
export FISOC_OM="ROMS"
export FISOC_OM_LIBS="-loceanM"
export FISOC_OM_INCLUDE="${MY_ROMS_DIR}/Build"
export FISOC_OM_LIBPATH="/usr/local/lib"
\end{lstlisting}

%ROMS uses C preprocessor (cpp) directives to determine which parts of the code are compiled, 
%and these may vary across experiments, requiring ROMS to be compiled multiple times. 
The ROMS cpp directives require matching cpp directives in the FISOC 
compilation process. 
These can be set for a FISOC build using the CPPFLAGS environment variable.
For example (see also the example FISOC build script):
\begin{lstlisting}
export CPPFLAGS="$CPPFLAGS -D ROMS_DDDT"
\end{lstlisting}
At the time of writing, the relevant values are ROMS\_MASKING, ROMS\_SPHERICAL, 
ROMS\_DDDT, \\
ROMS\_DSDT, ROMS\_DRAFT and ROMS\_AVERAGES. Use of the averaged melt rate from ROMS is activated in FISOC by the ROMS\_AVERAGES cpp. The general rule is that any of these directives that 
are defined for the ROMS compilation need to be defined also for the FISOC 
compilation. 




% put this in GMD paper?

%Setting FISOC\_DSDT activates the following in ROMS:

%In the same way that zice is added to the GRID type and used to indicate height of the lower surface of the iceshelf relative to sea level (i.e. iceshelf draft), sice is added to the GRID type to indicate height of the upper surface of the ice sheet/shelf. sice can be read from netcdf in the same was as zice.

%In the same way that iceshelf\_dddt is added to the ICESHELFVAR type and accessed from FISOC to set the rate of change of iceshelf draft wrt time, iceshelf\_dsdt is added to ICESHELFVAR to set the rate of change of sice with time.  If FISOC does not update iceshelf\_dsdt then the upper ice surface will remain static.

% The value for ice density (and a couple other things?) defined in iceshelf\_mod.h is used (overwrite these from FISOC?)


%\subsection{Troubleshooting}



\subsubsection{FVCOM}
\label{sec:PreReqFVCOM}

\textbf{Compiling FVCOM for FISOC} \\
FISOC has also been developed and tested with an ice shelf enabled version of unstructured grid Finite Volume Community Ocean Model as described in \url{https://doi.org/10.1016/j.ocemod.2019.101536}. General information about FVCOM can be found at \url{http://fvcom.smast.umassd.edu/fvcom/}. Note that this website contains a registration form which must be completed to download the model source code. The module for ice shelf cavity dynamics is available via a static patch (\url{https://doi.org/10.17632/m6g4c3hm9m.1}) that can be used to augment the standard version of the model, or via the development branch \url{https://source.coderefinery.org/apn/fvcom4_fisoc} (please contact the authors of the above work for access).


Example build scripts can be found in the directory of a git clone 
from the repository mentioned above (fvcom4\_fisoc). It is assumed that users are familiar with a standard FVCOM build process.  buildFVCOM\_FX4.sh is the build script for FISOC Example 4 and it calls buildFVCOM\_FX4.py that installs required libraries and compiles FVCOM. Any hard coded paths in buildFVCOM\_FX4.py should be replaced with corresponding paths to compile FVCOM for use with the Example 4.  
The build is specific to the machine on which 
FVCOM/FISOC was developed and tested.

The cpp options for building FVCOM are described in /fvcom4\_fisoc/FVCOM\_source/make.inc. The FISOC specific options are:
\begin{lstlisting}
 CPP_MISC ?= -DICESHELF -DICEDRAFTMORPHY -DFVCOM_API -DFISOC_DDDT  -DFISOC_DRAFT
\end{lstlisting}

Detailed description of the FISOC specific options is given below. \begin{flushleft}
To tell FVCOM to expect FISOC to provide the iceshelf change rate:\\
FISOC\_DRAFT\\
To tell FVCOM to expect FISOC to provide a cavity change rate:\\
FISOC\_DDDT\\
To tell FVCOM to expect FISOC to provide an upper ice surface change rate
(required for grounding line migration 
and is used to update the wet/dry masks based on floatation):\\
FISOC\_DSDT\\
To tell FVCOM to calculate the averages of the variables provided to FISOC at 
the end of each FVCOM run call:\\
FVCOM\_API\\
If ice shelf geometry evolution is required in FVCOM this options must be 
defined:\\
ICEDRAFTMORPHY\\
If grounding line movement is required in ROMS the following cpp options 
must be defined:\\
WET\_DRY
\end{flushleft}

\vspace{10pt}
\textbf{Compiling FISOC with FVCOM}\\
When compiling FISOC with FVCOM, FISOC needs to know where to 
find the relevant FVCOM libraries.  
This can be done at FISOC compile time through the 
\$FISOC\_OM
environment variables.  For example:

\begin{lstlisting}
export FISOC_OM="FVCOM"
export FISOC_OM_LIBS="-lfvcom_api -lmetis -ljulian"
export FISOC_OM_INCLUDE="${MY_FVCOM_DIR}/FVCOM_source"
export FISOC_OM_LIBPATH="/usr/local/lib"
\end{lstlisting}


% put this in GMD paper?

%Setting FISOC\_DSDT activates the following in ROMS:

%In the same way that zice is added to the GRID type and used to indicate height of the lower surface of the iceshelf relative to sea level (i.e. iceshelf draft), sice is added to the GRID type to indicate height of the upper surface of the ice sheet/shelf. sice can be read from netcdf in the same was as zice.

%In the same way that iceshelf\_dddt is added to the ICESHELFVAR type and accessed from FISOC to set the rate of change of iceshelf draft wrt time, iceshelf\_dsdt is added to ICESHELFVAR to set the rate of change of sice with time.  If FISOC does not update iceshelf\_dsdt then the upper ice surface will remain static.

% The value for ice density (and a couple other things?) defined in iceshelf\_mod.h is used (overwrite these from FISOC?)


%\subsection{Troubleshooting}






\section{Running FISOC}
\label{sec:FISOC_SUG}

The FISOC executable is by default called FISOC\_caller, and should be located in your 
path after installation. 
The installation is specific to the choice of component (you need to re-compile if you switch, for 
example, from one OM to another). 
%Unlike ROMS, you can compile FISOC with different choices in the same FISOC code folder as long as you define the FISOC executable with different names at each installation.  
Beyond choice of components, all run time choices are made in the FISOC\_config.rc file
(Section~\ref{sec:config}), 
or through component specific initialisation.

For example, you can run FISOC in serial like this:
\begin{lstlisting}
FISOC_caller
\end{lstlisting}
You can run FISOC in parallel like this (depending on your system):
\begin{lstlisting}
mpirun -np 4 FISOC_caller
\end{lstlisting}

In the first instance a dummy coupler can be run by setting both environment variables FISOC\_ISM and 
FISOC\_OM to ``dummy'' at compile time.  This can help to test the compilation, and was used during development, 
but performs no meaningful science.  

In verbose mode (Section~\ref{sec:config}) some run time information may be printed to the screen.  
Independently of this, log files are always written (see Section~\ref{sec:output}). 









\subsection{FISOC runtime configuration}
\label{sec:config}

The FISOC configuration file is named FISOC\_config.rc, and is expected to be present 
in the current directory when running FISOC.  
This is a resource file, as described by the 
ESMF documentation.  It supports different types and also lists. 
In principle, lists of mixed types are supported, though FISOC does not utilise this capability.
Basic syntax highlighting for .rc files can be activated within emacs 
(see notes in doc/FISOC\_emacsMode.asc in the FISOC repository).

FISOC\_config.rc mainly contains parameters specific to the coupling rather than to the running of 
individual components.  
Components should use their standard means for configuration, and component-specific configuration files 
should be specified in the FISOC configuration file.

Some of the FISOC config entries are strictly required and some are optional.
This section describes all the valid standard config entries. 
Note that model-specific non-standard entries can be added if needed, and use of these 
should be through the model-specific wrapper code.


\begin{flushleft}
%\textbf{label:}               [TYPE]   [Required?]                         \\
%Description                                                                \\
%\vspace{6pt}
%\vspace{6pt}
%\textbf{ISM\_meshFile:}       [STRING] [optional]                          \\
%The name of a netcdf file containing the ISM mesh in ESMF format.          \\
%\vspace{6pt}
\textbf{\underline{ISM options}} \\
\vspace{6pt}
\textbf{ISM\_configFile:}     [STRING] [optional]                          \\
The name of the ISM-specific config file.                                  \\
\vspace{6pt}
\textbf{ISM\_stdoutFile:}     [STRING] [optional]                          \\
The name of a file to which to write the ISM standard output.  May be 
required depending on model-specific wrapper.                              \\
\vspace{6pt}
\textbf{FISOC\_ISM\_ReqVars:} [STRING] [required]                          \\
List of variable names required to be provided by the ISM.                 \\
\vspace{6pt}
\textbf{ISM\_varNames:} [STRING] [optional]                                \\
List of native names of variables in the ISM, for use by the model-specific 
wrapper.  May also be required or unused depending on the wrapper. Must be 
same length as \textbf{FISOC\_ISM\_ReqVars}.                               \\
\vspace{6pt}
\textbf{FISOC\_ISM\_DerVars:} [STRING] [required]                          \\
List of variables derived by FISOC from the ISM variables. These are 
calculated from ISM required variables by hard coded routines in 
FISOC\_ISM or FISOC\_utils. This list is allowed to be empty.              \\
\vspace{6pt}
\textbf{ISM\_maskOMvars:} [LOGICAL] [optional]                             \\
Determines whether the ISM\_maskOMvars should be turned on or not. The generic
pre-requisite for this is that ISM\_gmask must be present, i.e. it must be a 
member of either FISOC\_ISM\_ReqVars or FISOC\_ISM\_DerVars.
The Elmer-specific pre-requisite is that the Elmer variable GroundedMask must
exist according to the .sif.                                               \\
\vspace{6pt}
%\textbf{ISM\_gridType:}        [STRING] [required]                         \\
%Which ESMF object to use.  Valid values are ESMF\_mesh (typically used 
%to describe unstructured meshes) and ESMF\_grid (typically used to 
%describe structured grids, supports stagered grids).                       \\
%\vspace{6pt}
\textbf{ISM2OM\_vars:}        [STRING] [optional]                          \\
List of variables to be passed from the ISM to the OM. Defaults to 
all ISM variables (union of required and derived variables). If an empty 
list is given no variables will be passed from the ISM to the OM.  This 
can be useful for unit testing.                                            \\ 
\vspace{6pt}
\textbf{ISM2OM\_init\_vars:}  [LOGICAL] [optional]                         \\
Determines whether the ISM2OM\_vars should be passed to the OM during the 
second phase of OM initialisation.   Default is .TRUE.                     \\ 
\vspace{6pt}
\textbf{ISM2OM\_regrid:}       [STRING] [optional]                         \\
The ESM regridding method to use.  If not set, the default will be 
written to the log files.  Possible values are given in the ESMF 
documentation. 
\url{http://www.earthsystemmodeling.org/esmf_releases/public/last/ESMF_refdoc/node9.html#SECTION090146000000000000000} \\
\vspace{6pt}
\textbf{ISM2OM\_extrap:}       [STRING] [optional]                         \\
Some ESMF regridding methods may be supplemented with an extrapolation
method for target points outside the source domain.
\vspace{6pt}

\vspace{16pt}

\textbf{\underline{OM options}} \\
\vspace{6pt}
\textbf{OM\_configFile:}      [STRING] [optional]                          \\
The name of the ISM-specific config file.                                  \\
\vspace{6pt}
\textbf{OM\_stdoutFile:}     [STRING] [optional]                           \\
The name of a file to which to write the OM standard output.  May be 
required depending on model-specific wrapper.                              \\
\vspace{6pt}
\textbf{FISOC\_OM\_ReqVars:}  [STRING] [required]                          \\
List of variable names required to be provided by the OM.                  \\
\vspace{6pt}
\textbf{OM\_ReqVars\_stagger:} [STRING] [optional]                         \\
Corresponding exactly to \textbf{FISOC\_OM\_ReqVars}, descriptions of the 
grid stagger for each variable.                                            \\
\vspace{6pt}
\textbf{FISOC\_OM\_DerVars:}  [STRING] [required]                          \\
List of variables derived by FISOC from the OM vars.  
To be calculated from OM vars by hard coded routines in FISOC\_OM.        \\
\vspace{6pt}
%\textbf{OM\_gridType:}        [STRING] [required]                           \\
%Which ESMF object to use.  Valid values are ESMF\_mesh and ESMF\_grid.     \\
%\vspace{6pt}
\textbf{OM2ISM\_vars:}        [STRING] [optional]                          \\
List of variables to be passed from the OM to the ISM. Defaults to 
all OM variables (union of required and derived variables). If an empty 
list is given no variables will be passed from the OM to the ISM.  This 
can be useful for unit testing.                                            \\ 
\vspace{6pt}
\textbf{OM2ISM\_init\_vars:}  [LOGICAL] [optional]                         \\
Determines whether the OM2ISM\_vars should be passed to the ISM during the 
second phase of ISM initialisation.   Default is .TRUE.                    \\ 
\vspace{6pt}
\textbf{OM\_initCavityFromISM:}  [LOGICAL] [optional]                      \\
Switch to allow the OM to overwrite its cavity geometry with ISM\_z\_l0 
during the second phase of initialisation.
Defaults to .FALSE.                                                        \\
\vspace{6pt}
\textbf{OM\_cavityUpdate:}   [STRING] [optional]                           \\
How to process ISM ice draft for use in OM.  Valid values are RecentIce    \\
(default), Rate, CorrectedRate, and Linterp.                               \\
\vspace{6pt}
\textbf{OM\_WCmin:}  [REAL] [optional]                                     \\
Minimum water column thickness imposed by OM.  Defaults to zero.  When 
using ROMS, this corresponds to the ROMS DCRIT (in the .in file) and 
should be set to the same value.  Only used with CorrectedRate to 
preserve a ``dry'' water column under grounded ice.                        \\
\vspace{6pt}
\textbf{OM\_CavCorr:}  [REAL] [optional]                                   \\
The proportion of the OM - ISM cavity discrepancy to correct in one 
FISOC call to the  OM run method. This cavity correction factor can take values from  0 to 1. Defaults to 0.2.  Only used with 
CorrectedRate.                                                             \\
\vspace{6pt}
\textbf{OM\_outputInterval:} [INTEGER][optional]                           \\
FISOC collects OM output once every OM\_outputInterval OM timesteps. 
Defaults to 1.  dt\_ratio/OM\_outputInterval must be integer.              \\
\vspace{6pt}
\textbf{OM2ISM\_regrid:}       [STRING] [optional]                         \\
The ESMF regridding method to use.  If not set, the default will be 
written to the log files.  Possible values are given in the ESMF 
documentation. 
\url{http://www.earthsystemmodeling.org/esmf_releases/public/last/ESMF_refdoc/node9.html#SECTION090146000000000000000}                                        \\
\vspace{6pt}
\textbf{OM2ISM\_extrap:}       [STRING] [optional]                         \\
Some ESMF regridding methods may be supplemented with an extrapolation
method for target points outside the source domain.                        \\
\vspace{6pt}

\vspace{16pt}


\textbf{\underline{Netcdf output options}}                                 \\
Note: these are considered OM options because the output writing occurs 
from the OM generic wrapper.  The OM import state here contains the 
ISM variables regridded to the OM grid.                                    \\
\vspace{6pt}
\textbf{OM\_writeNetcdf:}   [LOGICAL] [optional]                           \\
Switch for dumping the OM import and export variables to NetCDF files.
Defaults to .TRUE.                                                         \\
\vspace{6pt}
\textbf{output\_dir:}  [STRING] [optional]                                 \\
Path to directory (must already exist) to which to write the NetCDF files. 
Defaults to current directory.                                             \\
\vspace{6pt}
\textbf{OM\_NCfreq:}  [STRING] [optional]                                 \\
Output writing frequency.
Defaults to ``all''.  Valid values are ``all'' or ``ISM'' (only write 
netcdf outputs after an ISM timestep).
\vspace{22pt}

\textbf{\underline{Timestepping options}}                                  \\
\textbf{OM\_dt\_sec:}         [INTEGER][required]                          \\
OM timestep length in seconds.  When using ROMS, this may be a multiple 
of the ROMS timestep length, in which case each FISOC call to the ROMS 
run method will run multiple ROMS timesteps.                               \\
\vspace{6pt}
\textbf{dt\_ratio:}          [INTEGER][required]                           \\
ISM/OM timestep ratio.                                                     \\
\vspace{6pt}
\textbf{start\_year:}        [INTEGER][required]                           \\
Start year and month define the start time of the coupled simulation.      \\
\vspace{6pt}
\textbf{start\_month:}       [INTEGER][required]                           \\
\vspace{6pt}
\textbf{end\_year:}          [INTEGER][optional]                           \\
End year and month define the finish time of the coupled simulation.       \\
\vspace{6pt}
\textbf{end\_month:}         [INTEGER][optional]                           \\
\vspace{6pt}
\textbf{runLength\_ISM\_steps:} [INTEGER][optional]                        \\
As an alternative to specifying an end time, the run length in terms of 
ISM timesteps may be specified.                                            \\
\vspace{22pt}

%after adding atmos model...
%    "FISOC_dt_sec" "AM_ts" "OM_ts" "ISM_ts"
%    "AO_cpl_ts" "AIS_cpl_ts" "OIS_cpl_ts"

\textbf{\underline{General options}}                                       \\
\textbf{verbose\_coupling:}  [LOGICAL][required]                           \\
If true, some run time information is printed to the screen.  
A log file is always 
written, but writing to the log during timestepping is suppressed when 
verbose\_coupling is false.\\
\end{flushleft}



\subsubsection{FISOC variables}
\label{sec:FISOCvars}

The union of \textbf{FISOC\_ISM\_ReqVars}, \textbf{FISOC\_ISM\_DerVars}, \textbf{FISOC\_OM\_ReqVars} 
and \textbf{FISOC\_OM\_DerVars} describes the full set of variables required by FISOC for a given simulation. 
Valid values are given in Table~\ref{tab:vars}.
Note that the units given in Table~\ref{tab:vars} are suggested units.  FISOC doesn't care about units, but 
the user must ensure unit consistency.  There may be hard coded unit assumptions in the model specific 
wrappers.

\begin{table}
  \begin{center}
    \begin{tabular}{ l|l|l }
      Variable              & Description                                  & Units \\
      \hline
      ISM\_temperature\_l0  & Ice temperature at the ice ocean interface.  & K \\
      ISM\_temperature\_l1  & Ice temperature in the ISM one level above   & K \\
                            & the ice ocean interface.                     &   \\ 
      ISM\_z\_l0            & Height relative to sea level of the ice      & m \\
                            & base.                                        &   \\
      ISM\_z\_lts           & Height relative to sea level of the ice      & m \\
                            & top surface.                                 &   \\
      ISM\_z\_l1            & Height relative to sea level of the first    &   \\
                            & ISM model level above the ice base.          & m \\
      ISM\_z\_l0\_previous  & Height relative to sea level of the ice      & m \\
                            & base one ISM timestep previously.            &   \\
      ISM\_z\_lts\_previous & Height relative to sea level of the ice      & m \\
                            & top surface one ISM timestep previously.     &   \\
      ISM\_thick            & Ice thickness                                & m \\
      ISM\_dTdz\_l0         & Vertical temperature gradient in the ice     & K/m \\
                            & at the ice base.                             &   \\
      ISM\_dddt             & Rate of change of the ice draft with respect & m/a \\
                            & to time.                                     &   \\
      ISM\_dsdt             & Rate of change of the ice top surface        & m/a \\
                            & height with respect to time.                 &   \\
      ISM\_velocity\_l0     & Ice flow velocity at the ice base            & m/a \\
      ISM\_maskOMvars       & Mask the fields being passed from OM         &   \\
                            & to the ISM                                   &   \\
      OM\_bmb               & Ice shelf basal melt rate.                   & m/a \\
      OM\_temperature\_l0   & Ocean temperature at the ice ocean           &   \\
                            & interface.                                   &   \\
      OM\_z\_l0             & Height relative to sea level of the ice      & m \\
                            & base.                                        &   \\
      OM\_bed               & Ocean bathymetry                             &   \\
      OM\_z\_lts            & Height relative to sea level of the ice      & m \\
                            & top surface.                                 &   \\
    \end{tabular}
  \end{center}
  \caption{FISOC standard variables and typical units.  
    Note that heights relative to sea level are always positive upward.
    Some variables in this table differ only in their prefix (ISM or OM).
    The only differnce is that they are defined on the geometry object
    (grid or mesh)  of their respective component.
    For example ISM\_z\_lts is needed to pass the ice upper surface from the ISM
    to the OM, and OM\_z\_lts is needed in order to derive a drift correction if
    the correctedRate cavity option is used.
  }
  \label{tab:vars}
\end{table}



As a naming convention, ``z'' refers to the vertical coordinate, and ``l0'' and ``l1'' refer to the 
model level at the ice-ocean interface (this would typically be the lowest level of the ISM or 
the uppermost level of the OM) and one level above it, respectively. ``lts'' refers to the top 
surface of the ISM.

FISOC outputting occurs on the ocean grid, and consists of dumping both the import and 
export fields to netcdf files. 
\textbf{FISOC\_ISM\_ReqVars} may contain variables that are required only so that they can be written 
out on the ocean grid (typically as a sanity or regridding check) rather 
than actually being needed by the OM.

The list of derived variables, \textbf{FISOC\_ISM\_DerVars}, indicates which variables are needed by the 
OM but are not calculated by the ISM or its wrapper. 
The methods for calculating the derived variables are hard coded in FISOC\_ISM.f90. 
Valid values for  \textbf{FISOC\_ISM\_DerVars} include:

ISM\_z\_l0\_previous.  This is the depth of ice base at previous ISM timestep. This is simply stored 
in memory.  No calculation is required, but this variable is needed for the other ``derived'' variables. 

ISM\_dTdz\_l0.  Temperature gradient at ice base.  This is calculated as 
\begin{equation}
ISM\_dTdz\_l0 = \frac{ISM\_temperature\_l1 - ISM\_temperature\_l0}{ISM\_z\_l1 - ISM\_z\_l0}
\end{equation}

ISM\_dddt.  Rate of change of depth of ice base.  This is calculated as 
\begin{equation}
ISM\_dddt = \frac{ISM\_z\_l0 - ISM\_z\_l0\_previous}{ISM\_dt}
\end{equation}

ISM\_dsdt is calculated similarly to ISM\_dddt.

Not all ISM variables need to be passed to the OM, and vice versa.  
This choice is made by the 
user through use of configuration options ISM2OM\_vars and OM2ISM\_vars.  
These options alow the user to specify a subset of the full set of 
required and derived variables that will be passed to the other component 
in a given simulation.  An empty list can be used to avoid any variables 
being passed between components.  This can be useful during testing and 
troubleshooting.

A note on efficiency (May 2020): 
All the ISM and OM variables are currently being regridded and passed 
to the wrapper for the opposite component.  It is at the model-specific 
wrapper level that ISM2OM\_vars and OM2ISM\_vars are checked.  If regridding 
becomes a significant proportion of FISOC's computational cost this 
should be re-implemented to reduce non-essential regridding operations.





\subsubsection{Updating the ice shelf cavity for the OM}

Several options are available through FISOC for updating the OM representation 
of the ocean cavity. 
These vary from the simplest option of using 
the most recent cavity geometry from the ISM to update the OM representation of 
cavity geometry to smoother options via either interpolation in time or 
specifying a rate of change of cavity geometry.

The options (summarised in Table~\ref{tab:cavity}) are all implemented 
within FISOC, based on the cavity geometry 
calculated by the ISM.  
Some options are through FISOC derived variables, as described 
in Section~\ref{sec:FISOCvars}.  
The key ISM output from which all cavity options 
are calculated is ISM\_z\_l0.  

\begin{table}
  \begin{center}
    \begin{tabular}{ llll }
      Cavity option  & Summary                                    & Required ISM2OM \\
                     &                                            &  variable       \\
      \hline
      RecentIce      & Most recent ice draft from ISM             & ISM\_z\_l0      \\
      Rate           & Rate of change of ice draft from two most  & ISM\_dddt       \\
                     & recent ISM steps                           &                 \\
      CorrectedRate  & As above with additional drift correction  & ISM\_dddt       \\
      Linterp        & Time-interpolated ice draft from two most  & ISM\_z\_l0\_linterp \\
                     &  recent ISM steps                          &                  \\
    \end{tabular}
  \end{center}
  \caption{
    Cavity update options.  Note that of the possible ISM cavity variables 
    (ISM\_z\_l0, ISM\_z\_l0\_linterp, ISM\_dddt) only the required variable 
    (third column) should be passed to the OM (constrained using ISM2OM\_vars).
    At time of writing (May 2020) the Rate and CorrectedRate are giving 
    the most reliable outcomes and are recommended.
  }
  \label{tab:cavity}
\end{table}

The FISOC GMD paper will provide more  details about these approaches 
to updating the ocean 
representation of the ice shelf cavity, and also about the wetting and 
drying scheme used for grounding line migration.












\subsection{Timestepping}

Different asynchronous timestepping options are planned.
Currently recommended (May 2020) is to set dt\_ratio = 1 and use the 
ISM timestep for both the ISM and OM components. 
ROMS or FVCOM will run as many timesteps as needed  (see also  DT in the ROMS .in file) 
for each call to their run method. 
With this approach the Rate or CorrectedRate methods for cavity evolution 
are recommended. 
FISOC can pass the ROMS melt rate to the ISM after the call to the 
ROMS run method. 
%It might be preferable to get ROMS to provide an average rather than a final 
%snapshot for the melt rate. This should be straightforward but is NYI.

%***fill in this section we've implemented both tight coupling (ice and ocean both running 
%on the same timescale) and loose coupling (for longer time scales where the ocean is run 
%to steady state then the ice sheet continues until significant change has occurred in the cavity).




%\subsection{Parallelism}
%ESMF supports any combination of concurrent and or sequantial parallelisation. 
%FISOC is currently (6/12/2017) set up for sequential coupling, where the 
%full set of processors are used first for the ISM call then for the OM call. 
%It should not be too hard to implement something more flexible if needed 
%in the future.



\subsection{Running FISOC with Elmer/Ice} 
For dynamic linked libraries, shared object files may be needed at run time.  
This can be ensured through use of 
the \$LD\_LIBRARY\_PATH environment variable. 

For example (it is assumed \$ELMER\_HOME was set during Elmer installation):
\begin{lstlisting}
export LD_LIBRARY_PATH="$FISOC_ISM_LIBPATH/:$LD_LIBRARY_PATH"
\end{lstlisting}

As with a normal Elmer/Ice simulation, the mesh should be partitioned into the 
same number of partitions as the number of processors (which is the same as the number of 
ESMF PETs and DEs). 

More information about Elmer, and especially Elmer/Ice, can be found on several sources 
on the internet.

\begin{flushleft}
\url{https://www.csc.fi/web/elmer}\\
\url{http://www.nic.funet.fi/pub/sci/physics/elmer/doc/}\\
\url{http://elmerice.elmerfem.org/}\\
\url{http://elmerice.elmerfem.org/wiki/doku.php}\\
\url{http://www.elmerfem.org/forum/}\\
\end{flushleft}



\subsubsection{Elmer/Ice specific configuration}

The following options in the FISOC configuration file are used by Elmer/Ice.  These 
are non-standard configuration options.

\begin{flushleft}
\textbf{ISM\_BodyID:} [INTEGER] [required]                               \\
The body ID of the surface on which interactions with the ocean occurs.  
Typically this will be the lower surface, defined as a boundary in the   
Elmer/Ice mesh file and as a body in the boundary condition section of   
the .sif.                                                                \\
\vspace{6pt}
%\textbf{ISM\_ProjVector:} [INTEGER LIST] [optional]                      \\
%3D vector describing the view direction for use in node ordering of      
%elements.  Default down.  NYI                                            \\
%\vspace{6pt}
\end{flushleft}





\subsection{Running FISOC with ROMS}
\label{sec:runningROMS}

FISOC needs to access the shared library at run time.  One way of ensuring this 
is to add the location of the library to the \$ LD\_LIBRARY\_PATH variable, e.g.:
\begin{lstlisting}
export LD_LIBRARY_PATH="$FISOC_OM_LIBPATH/:$LD_LIBRARY_PATH"
\end{lstlisting}

ROMS writes a lot of information to the screen when run alone.  
When run through FISOC this is redirected to a text file. 
The file name is given by \textbf{OM\_stdoutFile}, 
which must be provided in the FISOC config file 
whenever ROMS is used.

The number of processes to launch FISOC with must be consistent with the number of 
partitions in the ROMS grid.  This is set in the \textbf{OM\_configFile} (i.e. the ROMS .in file) by the 
Ntile parameters.  For example, the following gives 4 partitions
\begin{lstlisting}
      NtileI == 1                               ! I-direction partition
      NtileJ == 4                               ! J-direction partition
\end{lstlisting}
%This could be launched with
%\begin{lstlisting}
%mpirun -np 4 FISOC_caller
%\end{lstlisting}

Some of the ROMS configuration information is in a .dat file.
When running ROMS through FISOC, the name and full path of this file must be given to VARNAME in the ROMS .in file.


% this stuff should now be covered by earlier sections:
%\subsubsection{ROMS preprocessor directives}
%Some aspects of the ROMS configuration are controlled through preprocessor directives.  
%ROMS must be recompiled if these are changed.  
%For example, there is a file located somewhere like 
%\begin{lstlisting}
%ROMS/Include/iceshelf2d.h
%\end{lstlisting}
%This file contains statements like this:
%\begin{lstlisting}
%#define ICESHELF
%#ifdef ICESHELF
%# undef ICESHELF_2EQN_VBC
%# define ICESHELF_3EQN_VBC
%# undef ICESHELF_TEOS10
%# undef ICESHELF_MORPH
%# define LIMIT_ICESTRESS
%#endif
%\end{lstlisting}
%In some cases changes are needed to be made to FISOC corresponding with these settings. 
%These are made to the hard coded values of variables defined at the head of the ROMS 
%wrapper module, contained in the file FISOC\_OM\_Wrapper\_ROMS.F90, with the 
%parameter  attribute. 
%Currently the following are in use, and further additions may be made as required.
%\begin{lstlisting}                                                                                           
%  LOGICAL, PARAMETER :: ROMS_MASKING = .FALSE.
%  LOGICAL, PARAMETER :: ROMS_SPHERICAL = .FALSE.
%\end{lstlisting}
%If these values need to be changed FISOC will of course need to be compiled again.






\subsection{FISOC output files}
\label{sec:output}

The text file  outputs from FISOC comprise: \\
1. Standard out from the job.  In interactive runs it is written to the screen. \\
2. Standard out from the ice model.  This is redirected to a file named in the FISOC\_config file. \\
3. Standard out from the ocean model.  This is redirected to a file named in the FISOC\_config file. \\
4. FISOC/ESMF log files (usually named PET*.log). These provide line numbers on which errors occurred.

FISOC can output basic netcdf files from the OM generic wrapper 
on the OM grid. 
FISOC can do this for both ISM and OM variables.
This capability is mainly recommended just for sanity checking.

%The ISM and OM have their standard out directed to files named in the 
%FISOC\_config.rc file. 
%At time of writing (5/12/2017) ROMS stdout is not correctly implemented and sometimes 
%writes to fort.31.

FISOC/ESMF log files have default filenames of ``PET\#.FISOC.Log'', where ``\#'' is a number from 0 upwards 
indicating the ``Persistent Execution Thread'' (PET). 
These logs are created by FISOC/ESMF and contain run time messages and errors that can 
be helpful with troubleshooting.
Note that by default FISOC appends to the logs rather than over-writing at run time, so you may wish to delete 
old logs periodically. 

Aside: PET is an ESMF abstraction designed to be general over differing parallel implementations. 
In FISOC, there is always a 1:1 relationship between PETs and MPI processes. 





\subsection{Troubleshooting}

If the error messages to the screen are not helpful, remember to check whether useful 
information has been logged in any of the text file outputs (Section~\ref{sec:output}).
In particular, the ESMF/FISOC log files contain line numbers in the 
code for all logged entries, including errors. 
These can be used to identify which code segment caused errors. 
%By default this will be in files in the current directory 
%with names like PETX.FISOC.Log (where X is a process number).
FISOC aims to provide sufficient troubleshooting information in these files.
However, not all problems are neatly reported at time of writing (May 2020).
Some examples of other errors are given below.

A segmentation fault has been known to occurr in the case of an incorrect path to the ROMS 
configuration file (the .in file). 

Note that DMUMPS error codes, should they occur, can be found in the MUMPS user guide.
MUMPS is often used by Elmer/Ice.
\url{http://mumps.enseeiht.fr/}

Errors like the following can ocurr (in the log files) when the number of processes is 
not consistent with the number of component partitons (this example involves ROMS):
\begin{lstlisting}
20151119 112603.491 ERROR            PET0 ESMCI_DistGrid.C:1200 ESMCI::DistGrid::create() Value unrecognized or out of range - deBlockList contains out-of-bounds elements
20151119 112603.491 ERROR            PET0 ESMCI_DistGrid_F.C:152 c_esmc_distgridcreatedb() Value unrecognized or out of range Internal subroutine call returned Error
20151119 112603.491 ERROR            PET0 ESMF_DistGrid.F90:1220 ESMF_DistGridCreateDB() Value unrecognized or out of range - Internal subroutine call returned Error
20151119 112603.491 ERROR            PET0 src/FISOC_OM_Wrapper_ROMS.f90:612 Value unrecognized or out of range - Passing error in return code
20151119 112603.491 ERROR            PET0 src/FISOC_OM_Wrapper_ROMS.f90:120 Value unrecognized or out of range - Passing error in return code
\end{lstlisting}

Errors like the following can occur if a component wrapper attempts to access a field 
that has not been created by FISOC, i.e. a field that is not in the list of
required variables in the FISOC config file (Section~\ref{sec:config}) 
(this example involves FISh):
\begin{lstlisting}
20151207 152613.483 ERROR            PET0 ESMCI_Container_F.C:165 ESMCI::Container::get() Invalid argument key does not exist
20151207 152613.484 ERROR            PET0 ESMCI_Container_F.C:448 c_esmc_containergetfield() Invalid argument Internal subroutine call returned Error
20151207 152613.484 ERROR            PET0 ESMF_Container.F90:589 ESMF_ContainerGetField() Invalid argument - Internal subroutine call returned Error
20151207 152613.484 ERROR            PET0 ESMF_FieldBundle.F90:1456 ESMF_FieldBundleGetItem() Invalid argument - Internal subroutine call returned Error
20151207 152613.484 ERROR            PET0 src/FISOC_ISM_Wrapper_FISh.f90:198 Invalid argument - Passing error in return code
\end{lstlisting}





\subsection{FISOC examples}

Example FISOC configurations can be found in the 
examples subdirectory of the repository. 

The ROMS setup for these examples is defined in the ROMS repository, not in the 
FISOC repository.  This is because some aspects of the ROMS setup are defined at 
compile time and it is standard ROMS development practice to use the ROMS 
repository for such details.  See below for specifics. 

The Elmer/Ice setup is defined in the FISOC 
repository.  
More information about running the examples can be found in 
examples/README.

In general it will be necessary to recompile ROMS and FISOC 
when switching between different examples, but it will not be necessary 
to recompile Elmer/Ice.

Examples  4 and 5 are described further in the FISOC GMD paper,
where they are referred to as Verification Experiments 1 and 2 (VE1 and VE2), 
along with presentation of outputs.
% 4 and 5 I think?



\subsubsection{Example 1: Long thin marine ice sheet}
A FISOC example using Elmer/Ice and ROMS. 
At time of writing (7/12/2017) this has not been 
recently used, may not work, and may be superceded by 
example 5. 
The Elmer/Ice setup is provided with the example in the subdirectory. 
The corresponding ROMS header file is ROMS/Include/iceshelf2d.h, 
the application is ICESHELF2D, and 
the input file is ROMS/External/ocean\_iceshelf2d.in.

\subsubsection{Example 2: Ice cliff}
Placeholder!  To be developed...

\subsubsection{Example 3: Using FISOC to handle time evolving forcing}
A FISOC example using offline forcing to drive ROMS. 
This is used for running the ISOMIP+ ocean 3 and 4 experiments.
The corresponding ROMS header file is ROMS/Include/isomip\_plus.h, 
the application is ISOMIP\_PLUS, 
and the input file is ROMS/External/ocean\_isomip\_plus\_ocn3.in.
Processed netcdf files based on those available through MISOMIP 
are also required.
The netcdf files are not included in the FISOC repository.


\subsubsection{Example 4: Simple floating only test}
A FISOC example using Elmer/Ice and ROMS. No grounding line is included. 
The Elmer/Ice setup is provided with the example in the subdirectory. 
The corresponding ROMS header file is ROMS/Include/iceshelf2d\_toy.h, 
the application is ICESHELF2D\_TOY, 
and the input file is ROMS/External/ocean\_iceshelf2d\_toy.in.
This example is used in the FISOC GMD model description paper where it is
referred to as Verification Experiment 1 (VE1\_ER).

\subsubsection{Example 5: Simple grounding line migration test}
A FISOC example using Elmer/Ice and ROMS. Similar domain to 
example 4, but with an evolving grounding line.
The Elmer/Ice setup is provided with the example in the subdirectory. 
The corresponding ROMS header file is ROMS/Include/iceshelf2d\_toy\_gl, 
the application is ICESHELF2D\_TOY\_GL, 
and the input file is ROMS/External/ocean\_iceshelf2d\_toy\_gl.in.
This example is used in the FISOC GMD model description paper where it is
referred to as Verification Experiment 2 (VE2\_ER).



\section{Post processing}

OM and ISM components should be able to provide their usual output formats,  
and the default expectation is that standard approaches to visualising and 
processing outputs will be used for OM and ISM components separately. 

The FISOC repository provides some limited functionality for output processing and
visualisation. 
The scripts provided are not robust, and are provided more as examples than as a
post processing or visualisation framework. 
These processing scripts are not supported by FISOC developers. 
A future intention  is to provide more robust and documented post processing
utilities. 

Elmer/Ice provides .vtu files that are viewable in Paraview.
ROMS provides netcdf files that can be made viewable in Paraview with a small amount
of manipulation.
The ROMS outputs need to be provided at rho points, and the vertical coords need to be processed to
express variables on depth levels instead of sigma coordinates. 
This way it is possible to view both ROMS and Elmer/Ice outputs together interactively through Paraview. 
This approach uses Netcdf Operators (NCO) and Matlab. 

ROMS Matlab repository information is available here:
https://www.myroms.org/wiki/Matlab\_Scripts.
You will need a ROMS account to access the matlab scripts in their subversion repository.
In particular set\_depth.m is needed if using the ROMS2para.m function provided
here (and set\_depth calls another ROMS matlab function...).

An example ncrcat command for processing ROMS output files is given here.
This command concatenates a subset of the ROMS output data into one netcdf file.
For example 5:
\begin{lstlisting} %[language=bash]

  ncrcat -O -p . -d ocean_time,,,2 -n 72,2,1 ocean_his_0001.nc ocean_his_select.nc -v ocean_time,x_rho,y_rho,Sb,Tb,draft,zeta,m,ubar_eastward,vbar_northward,w,u_eastward,v_northward,temp,salt,h,wetdry_mask_rho,Vtransform,Vstretching,theta_s,theta_b,hc

\end{lstlisting}

%For example 4 (without wet dry mask):
%\begin{lstlisting} %[language=bash]
%ncrcat -O -p . -d ocean_time,,,2 -n 72,2,1 ocean_his_0001.nc ocean_his_select.nc -v ocean_time,x_rho,y_rho,Sb,Tb,draft,zeta,m,ubar_eastward,vbar_northward,w,u_eastward,v_northward,temp,salt,h,Vtransform,Vstretching,theta_s,theta_b,hc
%\end{lstlisting}

%From NCO manual re "-n" option (we use this to determine which files to include):
%NCO decodes lists of such filenames encoded using the ‘-n’ syntax.  The simpler (three-argument) ‘-n’ usage takes the form -n file_number, digit_number, numeric_increment where file number is the number of files, digit number is the fixed number of numeric digits  comprising the numeric suffix,  and numeric increment is  the  constant,  integer-valued difference between the numeric suffix of any two consecutive files.

%From NCO manual re "-d" option (we use this to skip output steps for example if we want the same number of output steps in both ice and ocean):
%The stride is specified as the optional fourth argument to the ‘-d’ hyperslab specification: -d dim,[min][,[max][,[stride]]].  Specify stride as an integer (i.e., no decimal point) following the third comma in the ‘-d’ argument.


The concatenated file should be suitable for processing through the matlab scripts
that prepare for reading ROMS netcdf files into Paraview.

The ROMS2Para Matlab function is provided in the FISOC\_pp directory of the FISOC repository. 
This works for FISOC example 5, but may not work for other ROMS domains depending on the
coordinate system. 

\begin{table}
  \begin{center}
    \begin{tabular}{ l|l|l }
      \textbf{Script name}    & \textbf{Dependencies}       & \textbf{Function} \\
      \hline
      griddata\_fvcom.m       & Matlab                      & Regrid FVCOM output to regular grid. \\
      q\_read\_fvcom\_var.m   & Matlab, griddata\_fvcom.m   & Wrapper for griddata\_fvcom.m. \\
      ROMSvolume.py           & Python, Netcdf4 module      & Calculate ROMS total ocean volume. \\
      plotVols.m              & Matlab, ROMSvolume.py       & Line plots of Elmer, ROMS and FVCOM \\
                              &                             & volume or mass. \\
      read\_from\_roms.m      & Matlab                      & Read ROMS gridded data. \\
      velPlots.m              & Matlab, read\_from\_roms.m, & Comparative ocean velocity plot.  \\
                              & q\_read\_fvcom\_var.m       &  \\
      ROMS2Para.m             & Matlab                      & Prepare ROMS netcdf file for Paraview. \\
      ElmerGroundedArea.py    & Python, Paraview module     & Calculate Elmer/Ice total grounded   \\
                              &                             & area over time. \\
      IntegrateMelt.py        & Python, Paraview module     & Integrated Elmer/Ice total basal melt \\
                              &                             & over time.  \\
      ROMSgroundedArea.py     & Python, Paraview module     & Calculate ROMS total dry cell area  \\
                              &                             & over time. \\ \\
    \end{tabular}
  \end{center}
  \caption{
    Summary of example output processing scripts.
    These are not intended to be robust and are not documented.
    They are provided only as examples. 
  }
  \label{tab:ppscripts}
\end{table}

There are also some example scripts using either Matlab or the Paraview simple module or the Netcdf4 module
for Python to perform certain specific postprocessing tasks.
These are not documented but are summarised in Table~\ref{tab:ppscripts}. 



%It is expected that the OM and ISM components will output data in their usual formats, 
%and that this will form the basis for most output visualisation and post-processing.
%FISOC does however have the capcity to output both ISM and OM fields on the OM grid 
%in netcdf format.

%Scripts for visualising these outputs may be developed at some point, but no firm 
%plans exist.  If/when developed, they will be located in the FISOC\_pp directory.
%There is currently (end 2016) an inflexible script and unfriendly script used as a sanity 
%check during development.

%The script requires a recent python installation and the netcdf4-python module.


%\subsection{Netcdf4-python installation notes}

%Depending on your python installation, you may need to run something like this
%before installing netcdf4-python:
%\begin{lstlisting}
%sudo apt-get install python-dev
%\end{lstlisting}
%More information about the netcdf4-python module can be found here:\\
%\url{http://unidata.github.io/netcdf4-python/}\\

%You can clone or download the latest netcdf4-python code from the GitHub 
%repository:
%\begin{lstlisting}
%git clone git@github.com:Unidata/netcdf4-python.git 
%\end{lstlisting}
%Then build and install the netcdf4-python library like this:
%\begin{lstlisting}
%python setup.py build
%python setup.py install
%\end{lstlisting}
%Install may require root privileges.
%You may need to restart your shell.  

%A minor discrepancy between the python setup script and the way nc-config 
%retrieves relevant flags may lead to errors if mpi wrappers are used. 
%This issue should probably be considered a bug in nc-config. 
%More information (and a dirty fix) can be found here:\\
%\url{https://github.com/Unidata/netcdf4-python/issues/491}\\






\section{FISOC design and development}
\label{sec:FISOC_SDG}

This section describes aspects of FISOC design/development that an end user
would typically not need to know.
An overview of FISOC code modules is summarised
in Table~\ref{tab:modules} and this information is presented visually 
in Figure~\ref{fig:codeStruct}.
FISOC is intended to be flexible from top down and bottom up:
FISOC could be called as part of a larger ESMF coupled model by interfacing to FISOC\_parent
as a shared library (as of May 2020 the code is not correctly distributed between caller and parent;
please contact the developers if you wish to be able to call FISOC as an ESMF component and we
can address this issue).
FISOC calls independent ice or ocean components through model specific wrappers. 

\begin{table}
  \begin{center}
    \begin{tabular}{ l|l }
      \textbf{Generic module} &  \textbf{Role}                              \\
      \hline
      FISOC\_caller.f90     &  The FISOC calling program \\
      FISOC\_parent.f90     &  FISOC's top level ESMF gridded component \\
      FISOC\_coupler.f90    &  ESMF coupler component, handles regridding \\
      FISOC\_ISM.f90        &  ESMF gridded component, top level control for ISM \\
      FISOC\_OM.f90         &  ESMF gridded component, top level control for OM  \\
      FISOC\_Utils.f90      &  Assorted FISOC utilities, available to all modules \\
      \\ 
      \textbf{Component-specific module} &                       \\
      \hline
      FISOC\_OM\_Wrapper\_XXX.f90  & Model-specific wrapper for OM XXX \\
      FISOC\_ISM\_Wrapper\_XXX.f90 & Model-specific wrapper for ISM XXX \\
    \end{tabular}
  \end{center}
  \caption{
    Summary of the role of key FISOC code modules.
    ISM is short for Ice Sheet Model and OM is short for Ocean Model.
  }
  \label{tab:modules}
\end{table}

\begin{figure}[t]
  \vspace*{2mm}
  \begin{center}
    \includegraphics[width=17cm]{FISOC_structure2.pdf}
%    \includegraphics[width=17cm]{FISOC_structure.pdf}
  \end{center}
  \caption{FISOC code and data structures.}
  \label{fig:codeStruct}
\end{figure}

The FISOC\_ISM and FISOC\_OM modules provide top level control and processing for ice and ocean
components.
These are not model-specific.
The model specific code, which exchanges fields between 
ESMF structures and structures used by the individual models, is in the FISOC\_(O/IS)M\_Wrapper modules.
The main ISM or OM call structure must be made to be compatible with ESMF, which essentially means 
having seperate initialise, run and finalise calls.  See the ESMF documentation.





\subsection{ESMF run time objects}

FISOC maps component grids or meshes to ESMF\_grid or ESMF\_mesh objects within the model
specific wrappers. 
FISOC also maps component ``variables''  or ``fields'' to ESMF\_field objects
in the model specific wrappers.
These objects contain pointers to the ESMF\_grid or ESMF\_mesh objects. 
Multiple  ESMF\_field objects are wrapped in ESMF\_fieldbundle objects.

ESMF\_state objects store all data for a gridded component.
The model-specific wrappers do not use this level of ESMF code abstraction.
Instead the field bundles are packed/unpacked to/from state objects
within the non model-specific wrappers
(i.e. FISOC\_OM and FISOC\_ISM modules).

The ESMF\_routeHandle objects are used to store weights and all information
needed for regridding
operations. 
The coupler component within FISOC sets these up during initialisation using
ESMF\_mesh or ESMF\_grid objects obtained from the OM and ISM components. 

The ESMF virtual machine (ESMF\_VM) object provides a generic parallel context
and can contain information for multiple possible parallelism paradigms.
ESMF persistent execution threads (PETs) are also generic objects, representing
individual threads within the parallel context.
FISOC does not aim for this level of generality.
FISOC is intended to use only the message passing interface (MPI), and
requires a one to one mapping between MPI processes, PETs and domain partitions
within geometry objects. 
FISOC uses ESMF methods to intialise the parallel context.
FISOC then extracts an MPI communicator from the ESMF\_VM object and passes this to the
ISM and OM components to use in their initialisation phases.

FISOC also uses EMSF objects for time keeping, but these are not discussed here.



\subsection{Incorporating new OM or ISM components}

Structural changes to FISOC would be needed to introduce a new type 
of component, e.g. an atmosphere model.
Implementing an alternative ISM or OM should require no changes 
to existing FISOC code, just an additional model-specific wrapper.

Any new OM or ISM component to be used with FISOC must first be ESMF compliant.  This basically 
means that it should have an initialise, run and finalise routine, and that the developer can 
provide the component's grid and variables in ESMF compatible structures at run
time through the new wrapper.
The ESMF web site provides further documentation.
\url{https://www.earthsystemcog.org/projects/esmf/}


The new wrapper must contain a Fortran module with restrictions on the module name 
and on the  interface for the initialise, run and finalise methods. 
The precise requirements are not documented here as they may evolve 
during continued development.  
It is recommended to view an existing wrapper, especially the 
PUBLIC routines.

If it is found that changes to other aspects of the FISOC code are required, this should be 
implemented in collaboration with the FISOC core developers.





\subsection{Coding practices}

A new component wrapper  should be in a Fortran 90 module.  
All modules should contain the ``implicit none'' statement at the top (immediately after any 
``USE'' statements).  This property will be inherited by all procedures in the module.

FISOC modules have the private attribute, with only required procedures being 
made public. 
New model-specific wrapper modules should ensure that the initialise, run and finalise 
subroutines are public. 

Please implement your developments in a new branch in the FISOC repository. 
Request a merge to master when you are happy that it is stable. 

Adding new ISM or OM components should not require changes to the Makefile.
If you feel that changes to the Makefile are needed, please first contact the developers
to discuss whether this can be achieved through a 
build script.


\subsubsection{Error handling}

ESMF provides defensive error handling, with error codes and error messages passed up the 
call stack. 
FISOC implements ``fail-fast'' error handling, with errors generally being considered 
fatal. 
Exceptions to this may be made where it is safe to do so (e.g. where a default value can be 
used in the event of  failure to find a config parameter).
All calls to ESMF routines have return codes checked immediately, and errors logged.
If components (OM or ISM) provide a return code or error code, 
this should be checked by the component wrapper.

Note that many FISOC subroutines contain a return code, but these are mostly not used in 
practice.  Since errors are generally considered instantly fatal in FISOC, execution will 
generally not get as far as returning a failure code. 
If a more defensive rather than fail-fast approach is adopted in the future these return 
codes can be used.



\subsection{Configuration options}

Compile time choices include which ISM or OM component to run, the type of ESMF
geometry object for each component (mesh or grid), and paths to find
component libraries.
These are set through environment variables during the build process.
Other options are mostly run time options, determined by parameters in the FISOC
configuration file. 

The configuration file must be named ``FISOC\_config.rc''.  
It is compatible with ESMF config methods.  
An ESMF\_config object is automatically created from this file at run time.
This object is always named ``FISOC\_config'' in the FISOC code.
New model-specific parameters needed by the new wrappers may be 
introduced to the config file and can be accessed 
within the model-specific wrapper at run time, via  ``FISOC\_config''. 
No further coding is needed for this functionality.
See existing model-specific wrappers (e.g. FISOC\_ISM\_Wrapper\_Elmer.f90) 
for examples of this.

The master list of standard FISOC configuration options is contained in 
FISOC/doc/FISOC\_emacsMode.asc and in Section~\ref{sec:config} of this 
manual. 
When new standard configuration options are added during development, they 
should also be added to both the emacsMode file and  
Section~\ref{sec:config} of the manual.


\subsubsection{Default values and derived attributes}

The config file is intended to avoid duplication of information.  
In the case of configuration options that can be derived from other 
configuration options, their derivation is hard coded into 
FISOC\_utils.f90 (see FISOC\_ConfigDerivedAttribute interface) 
rather than adding redundant parameters to the config file.
It is recommended (though not strictly required) that further developments 
also follow this approach. 

Note that the approach of derived attributes can be used to hard code a 
default value for an attribute that is in essence not a derived attribute. 
Using the derived attribute subroutines as a wrapper in which to hard 
code a default value has the benefit of entering the hard coded value 
only at one location rather than each time the attribute is accessed 
from the config object.
It has the disadvantage that the utils module is not the most intuitive
place to store hard coded defaults.
Where defaults are set by the utils module, an info statement should be written
to the PET logs. 

The derived attributes and default values are not fully documented here due to the
dynamic nature of the code base; the code base should be considered to contain the definitive
list.
Some indication can be obtained by searching the code.
This command, for example, shows the lines surrounding where FISOC catches a ``not found''
error from ESMF: 
\begin{lstlisting}[language=bash]
 grep -i -C 3 ESMF_RC_NOT_FOUND FISOC_utils.f90
\end{lstlisting}
Note that it is not intended for physical parameters to have hard coded defaults given
using this mechanism.
It is dangerous to provide values for physical parameters in more than one location.
Use of derived attributes to set defaults is aimed at safe parameters e.g. output frequency.

%\subsection{ISM wrapper}

%\subsection{OM wrapper}

%\section{Future developments}



\subsection{Sequential parallelism}

\begin{wraptable}{r}{6cm}
  \begin{tabular}{|l}
    \textbf{FISOC component event order}  \\
    \hline
    OM intialisation phase 1 \\
    ISM intialisation phase 1 \\
    Coupler intialisation phase 1 \\
    OM intialisation phase 2 \\
    Coupler intialisation phase 2 \\
    ISM intialisation phase 2 \\
    OM run \\
    Coupler run phase 1 \\
    ISM run \\
    Coupler run phase 2 \\
    OM finalise \\
    ISM finalise \\
    Coupler finalise \\
  \end{tabular}
%  \caption{
%  }
  \label{tab:order}
\end{wraptable}

ESMF supports sequential and concurrent parallel coupling, and any combination of these. 
FISOC currently (May 2020) supports only sequential coupling, and requries that
the ISM and OM alternate use of the same set of processors.
This is implemented by duplicating the MPI context from the ESMF virtual machine (VM)
and passing this to each component. 
This can be made more flexible in the future if needed. 

%Table~\ref{tab:order}
The Table to the right summarises the order of events within this sequential coupling paradigm.
This ordering is hard coded, but again, this could be made more flexible in the future
if needed.
Note that there are two initialisation phases so that the ISM and OM components have the chance
to modify their initial state based on the initial state of the other component
(e.g. the OM may wish to update its ice shelf cavity geometry directly from the initialised
ISM during OM intialisation phase 2).
Note that the run phases are repeated as many times as are required to complete the simulation.
The coupler component  has multiple phases in order to separately handle
the regridding in each direction (ice to ocean and ocean to ice).
ESMF places no limitation on the number of initialisation or run phases for each component,
although FISOC currently imposes this fixed ordering, hard coded into FISOC\_parent.




\clearpage

\appendix

\section{Pre-requisite installation notes}
\label{app:A}
The following commands worked to install NetCDF and ESMF on a Linux Mint system in 2015 
in a suitable configuration for use with FISOC. 
Some pre-requisites for netcdf were also installed.
The system already had a working OpenMPI installation.
The following is just an example, it is not intended as a usable script. 


\begin{lstlisting}[language=bash]


# instructions on installing ESMF can be found here:
# http://www.earthsystemmodeling.org/esmf_releases/last_built/ESMF_usrdoc/node9.html

# netcdf instructions
# http://www.unidata.ucar.edu/software/netcdf/docs/netcdf-install.html

cd /somewhere/to/download/and/compile/source/code

sudo apt-get install m4

wget ftp://ftp.unidata.ucar.edu/pub/netcdf/netcdf-4/zlib-1.2.8.tar.gz
tar -xzf zlib-1.2.8.tar.gz 
cd zlib-1.2.8
 ./configure --prefix=/usr/local/
 make check
 sudo -E make install
cd ..

wget ftp://ftp.unidata.ucar.edu/pub/netcdf/netcdf-4/hdf5-1.8.13.tar.gz	
tar -xzf hdf5-1.8.13.tar.gz 
cd hdf5-1.8.13
 # Note the O0 flag in the next line.  The default is O3, 
 # which is strong optimisation.  This can result in failed 
 # checks on some systems.
 CFLAGS="-O0 -fPIC" CC=mpicc CXX=mpiCC FC=mpif90 ./configure --prefix=/usr/local/ --with-zlib=/usr/local  --enable-fortran --enable-parallel --enable-shared
 make check
 sudo -E make install
cd ..

# note that netcdf fortran library is now compiled from a 
# seperate source from the main netcdf c library. Install 
# the c library first, and make sure to create the shared
# object file. 
wget ftp://ftp.unidata.ucar.edu/pub/netcdf/netcdf-4.3.3.tar.gz
tar -xzf netcdf-4.3.3.tar.gz 
cd netcdf-4.3.3/
 LIBS=-ldl CC=mpicc CXX=mpiCC FC=mpif90 CPPFLAGS=-I/usr/local/include/ LDFLAGS=-L/usr/local/lib/  ./configure --prefix=/usr/local --enable-parallel 
 make check
 sudo -E make install
cd ..

wget ftp://ftp.unidata.ucar.edu/pub/netcdf/netcdf-fortran-4.4.2.tar.gz
tar -xzf netcdf-fortran-4.4.2.tar.gz 
cd netcdf-fortran-4.4.2
 LIBS=-ldl CC=mpicc CXX=mpiCC FC=mpif90 LDFLAGS=-L/usr/local/lib/ CPPFLAGS="-I/usr/local/include -DUSE_NETCDF4"  ./configure --prefix=/usr/local
 make check
 sudo -E make install
cd ..

# convenient viewer for contents of netcdf files (not essential)
sudo apt-get install ncview

# ESMF requires ESMF_DIR and probably other environment variables.  
# These can be set at the command line or, for example, in your 
# .bashrc fileor a local script file.  These might work:
export ESMF_DIR="/top/level/directory/for/esmf/"
export ESMF_NETCDF="split"
export ESMF_NETCDF_INCLUDE="/usr/local/include"
export ESMF_NETCDF_LIBPATH="/usr/local/lib"
export ESMF_COMM="openmpi"
export ESMF_PIO="internal"
                                                                                                              
wget downloads.sourceforge.net/project/esmf/ESMF_6_3_0r/ESMF_6_3_0rp1/esmf_6_3_0rp1_src.tar.gz
tar -xf esmf_6_3_0rp1_src.tar.gz
cd esmf 
 make check
 sudo -E make install
cd ..

# In order to actually use ESMF you must set the environment 
# variable ESMFMKFILE.  If you didn't use environment 
# variables to specify the install location this make file 
# will probably end up somewhere like this:
export ESMFMKFILE="$ESMF_DIR/DEFAULTINSTALLDIR/lib/libO/Linux.gfortran.64.openmpi.default/esmf.mk"

\end{lstlisting}
\vspace{10mm}
... or for ESMF version 7.0.0 ...
\begin{lstlisting}[language=bash]
wget https://sourceforge.net/projects/esmf/files/ESMF_7_0_0/esmf_7_0_0_src.tar.gz
tar -xf esmf_7_0_0_src.tar.gz 
\end{lstlisting}
\vspace{10mm}
... or for ESMF version 7.1.0 beta snapshot 14 ...
\begin{lstlisting}[language=bash]
git archive --remote=git://git.code.sf.net/p/esmf/esmf --format=tar --prefix=esmf/ ESMF_7_1_0_beta_snapshot_14 | tar xf -
\end{lstlisting}

\end{document}
