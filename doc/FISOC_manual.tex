
\documentclass[12pt]{article}
%\documentclass[12pt,twocolumn]{article}

\usepackage{listings}
\usepackage{color}
\usepackage{hyperref}


\definecolor{dkgreen}{rgb}{0,0.6,0}
\definecolor{gray}{rgb}{0.5,0.5,0.5}
\definecolor{mauve}{rgb}{0.58,0,0.82}

\lstset{%frame=tb,
  language=bash,
  aboveskip=3mm,
  belowskip=3mm,
  showstringspaces=false,
  columns=flexible,
  basicstyle={\small\ttfamily},
  numbers=none,
  numberstyle=\tiny\color{gray},
  keywordstyle=\color{blue},
  commentstyle=\color{dkgreen},
  stringstyle=\color{mauve},
  breaklines=true,
  tabsize=3, 
  postbreak=\raisebox{0ex}[0ex][0ex]{\ensuremath{\color{red}\hookrightarrow\space}}
}

\begin{document}

\title{Framework for Ice Sheet - Ocean Coupled modelling (FISOC) Manual}
\author{Rupert Gladstone (RupertGladstone1972@gmail.com) \and Ben Galton-Fenzi}
\date{Version 0.1, September 2015}
\maketitle

\newpage 
\tableofcontents
\newpage 

\section{Introduction}

The ``Framework for Ice Sheet - Ocean model Coupling" (FISOC) has been written to enable Ice Sheet Models 
(ISMs) and Ocean Models (OMs) to be run as a single executable to address the co-evolution of ice and ocean 
properties.  it is primarily designed to deal with the situation of a floating ice shelf at the interface 
between land ice and open ocean.

In this context an ISM simulates (part of) a marine ice sheet, including both grounded and floating parts, 
representing the dynamic evolution over time of the ice sheet.

An OM simulates the sub-shelf cavity circulation under the floating part of the ice sheet, and optionally also 
a wider ocean domain.

FISOC is a set of code modules and driver built using the Earth System Modelling Framework (ESMF, 
\url{https://www.earthsystemcog.org/projects/esmf/}). 
Some knowledge of the ESMF is essentially in order to fully understand the FISOC code.  It should 
be possible to run FISOC without knowledge of ESMF.

A full description of the physical processes FISOC attempts to simulate are provided in (***
ref GMD paper, not yet written).  
This manual describes how to use FISOC with an ISM and OM for which it has been 
developed (Section~\ref{sec:FISOC_SUG}) and also how to build additional ISM or OM components 
into the FISOC framework (Section~\ref{sec:FISOC_SDG}).




\section{Installing FISOC with established components}
\label{sec:FISOC_install}

FISOC can be obtained by emailing a request to the author.  It is maintained and developed in a 
private GitHub repository.  We intend to make this public in 2016.

FISOC has a simple build process.  The Makefile in the top level FISOC directory contains the 
hard coded dependencies needed to build FISOC code.  The Makefile refers to certain 
environment variables to determine paths and component choices (Section~\ref{sec:EnvVars}). 

Having installed the pre-requisites (Section~\ref{sec:PreReq}), simply run at the command line, 
in the top level FISOC directory:
\begin{lstlisting}
make install
\end{lstlisting}

An example script to build and install FISOC, 'buildFISOCexample.sh', is available in the top 
level FISOC directory.

FISOC has been tested with GNU compilers on Linux Mint using OpenMPI. 



\subsection{FISOC Environment Variables}
\label{sec:EnvVars}

Several environment variables may be used in the build process. 
Some of these are mandatory. 
Variables listed as optional here may be mandatory for configuratons other than 
``dummy''.
These environment variables are not used at run time, only during 
the compilation/installation of FISOC.

\begin{flushleft}
\textbf{FISOC\_INSTALL\_DIR}                       \\ 
Optional.                                          \\
Determines where FISOC\_caller will be installed. Defaults to \$HOME/bin. 
The user should also ensure this location is in their \$PATH. \\
\vspace{6pt}
\textbf{FISOC\_MPI}                                \\ 
Optional. Possible value ``yes''.                  \\
Sets preprocessor flag to tell FISOC whether this is a parallel compilation.
Defaults to serial.                                \\
\vspace{6pt}
\textbf{FISOC\_OM}                                 \\ 
Required. Possible values ``dummy'', ``ROMS''.     \\
Determines which OM component will be used.        \\
\vspace{6pt}
\textbf{FISOC\_OM\_INCLUDE}                       \\ 
Optional.                                          \\
Specifies the path where the OM header files are located.\\
\vspace{6pt}
\textbf{FISOC\_OM\_LIBPATH}                       \\
Optional.                                          \\
Specifies the path where the OM library files are located.\\
\vspace{6pt}
\textbf{FISOC\_OM\_LIBS}                          \\
Optional.                                          \\
Specifies the linker directives needed to link the OM library to FISOC. \\
\vspace{6pt}
\textbf{FISOC\_ISM}                                \\
Required. Possible values ``dummy'', ``Elmer''.    \\
Determines which ISM component will be used.       \\
\vspace{6pt}
\textbf{FISOC\_ISM\_INCLUDE}                       \\ 
Optional.                                          \\
Specifies the path where the ISM header files are located.\\
\vspace{6pt}
\textbf{FISOC\_ISM\_LIBPATH}                       \\
Optional.                                          \\
Specifies the path where the ISM library files are located.\\
\vspace{6pt}
\textbf{FISOC\_ISM\_LIBS}                          \\
Optional.                                          \\
Specifies the linker directives needed to link the ISM library to FISOC. \\
\end{flushleft}



\subsection{Pre-requisites}
\label{sec:PreReq}

A Message Passing Interface (MPI) implementation, such as 
OpenMPI (\url{http://www.open-mpi.org/}), is required. 

The Network Common Data Form (NetCDF,  
\url{http://www.unidata.ucar.edu/software/netcdf/}) 
Fortran interface must be available 

ESMF must be available (\url{https://www.earthsystemcog.org/projects/esmf/}).  
ESMF should have been built with NetCDF and MPI 
(see also notes in Appendix~\ref{app:A}).

%For example, environment variables like these may be used for the ESMF build
%\begin{lstlisting}
%export ESMF_NETCDF="split"
%export ESMF_NETCDF_INCLUDE="/usr/local/include/"
%export ESMF_COMM="openmpi"
%\end{lstlisting}

Viable ISM and OM components must be available for any physically meaningful simulations
(the build may be tested using ``dummy'' components).  
See Sections~\ref{sec:PreReqElmer} and \ref{sec:PreReqROMS}.

\subsubsection{Elmer/Ice}
\label{sec:PreReqElmer}

***how this should be written depends on whether the FISOC changes to 
Elmer code make it in to the Elmer repository. currently FISOC contains 
top level control structures that replace Elmer ones and the whole of 
Elmer needs to be re-compiled.  We aim to commit these changes to the 
main Elmer/Ice repo.***

***ref and link Elmer/Ice

When compiling FISOC with Elmer/Ice, FISOC needs to know where to 
find the relevant Elmer/Ice libraries.  
This can be done at FISOC compile time through the 
\$FISOC\_ISM
environment variables.  


\begin{lstlisting}
export FISOC_ISM="Elmer"
export FISOC_ISM_INCLUDE="$ELMER_HOME/share/elmersolver/include"
export FISOC_ISM_LIBPATH="$ELMER_HOME/lib/"
export FISOC_ISM_LIBS="-lelmersolver"
\end{lstlisting}





\subsubsection{ROMS}
\label{sec:PreReqROMS}

FISOC has been developed and tested with an ice shelf enabled version of ROMS. 
This is branched from the Rutgers ROMS repository.  Information about the Rutgers ROMS 
can be found at \url{https://www.myroms.org/}.

Development of the ice shelf enabled version  is currently ongoing in a private repository 
(please contact the developers if you need access to this).
A public version of the ice shelf enabled version is also available through github: 
\url{https://github.com/bkgf/romsIceShelf}.  
Please use the "FISOC\_friendly" branch.

***I need to merge the deve version to the public version before releasing this!!!***

Example build scripts can be found in the ROMS/Bin subdirectory of a git clone 
from the repository mentioned above.
It is now assumed that users are familiar with a standard ROMS build process. 

When compiling ROMS for use with FISOC, the following additional environment
variables are needed:
\begin{lstlisting}
export MAKE_SHAREDLIB="Yes"
export LIBDIR="/usr/local/lib"
export MY_CPP_FLAGS=" -DFISOC"
\end{lstlisting}

It is essential to activate the option to compile the ROMS shared library, which 
is done by setting the environment variable MAKE\_SHAREDLIB to any value. 

The shared library will be installed in the location given by the 
 \$LIBDIR environment variable. 

The -DFISOC flag activates preprocessor options.  This includes telling ROMS to use a 
specific hard coded unit rather than outputting to screen.  This relies on the 
same unit being hard coded in the FISOC ROMS wrapper, and results in the ROMS 
messages being sent to file instead of printed to screen.

By default ROMS will install the module files in the directory given by 
 \$SCRATCH\_DIR.  

When compiling FISOC with ROMS, FISOC needs to know where to 
find the relevant ROMS libraries.  
This can be done at FISOC compile time through the 
\$FISOC\_OM
environment variables.  For example:

\begin{lstlisting}
export MY_ROMS_DIR="/home/elmeruser/Source/ROMSIceShelf_devel"
export FISOC_OM="ROMS"
export FISOC_OM_LIBS="-loceanM"
export FISOC_OM_INCLUDE="${MY_ROMS_DIR}/Build"
export FISOC_OM_LIBPATH="/usr/local/lib"
\end{lstlisting}






\subsection{Troubleshooting}







\section{Running FISOC}
\label{sec:FISOC_SUG}

The FISOC executable is called FISOC\_caller, and should be located in your path after installation. 
The installation is specific to the choice of component (you need to re-compile if you switch, for 
example, from one ISM to another).  
Beyond choice of components, all run time choices are made in the FISOC\_config.rc file
(Section~\ref{sec:config}), 
or through component specific initialisation.

In the first instance a dummy coupler can be run be setting both environment variables FISOC\_ISM and 
FISOC\_OM to ``dummy'' at compile time.  This can help to test the compilation, and was used during development, 
but performs no meaningful science.  

In verbose mode (Section~\ref{sec:config}) some run time information may be printed to the screen.  
Independently of this, log files are always written, 
with default filenames of ``PET\#.FISOC.Log'', where ``\#'' is a number from 0 upwards indicating the 
``Persistent Execution Thread'' (PET). 
These logs containing run time messages and errors can be helpful with troubleshooting.
Note that FISOC appends to the logs rather than over-writing at run time, so you may wish to delete old logs 
periodically. 

Aside: PET is an ESMF abstraction designed to be general over differing parallel implementations. 
In FISOC, there is always a 1:1 relationship between PETs and MPI processes. 
%(Section~\ref{sec:FISOC_SDG}). 


\subsection{FISOC configuration}
\label{sec:config}

The FISOC config file is named FISOC\_config.rc, and is expected to be present 
in the current directory.  
This is a resource file, as described by the 
ESMF documentation.  It supports different types and also lists. 
In principle, llists of mixed types are supported, though FISOC does not utilise this capability.

FISOC\_config.rc contains code specific to the coupling.  Components may also use their 
independent means for configuration.

Some of the FISOC config entries are required.
This section describes all the valid standard (but note that model-specific non-standard entries 
can be added as required) entries in a FISOC config file as follows:

\begin{flushleft}
\textbf{label:}               [TYPE]   [Required?]                         \\
Description                                                                \\
\vspace{6pt}
\vspace{6pt}
\textbf{ISM\_meshFile:}       [STRING] [optional]                          \\
The name of a netcdf file containing the ISM mesh in ESMF format.          \\
\vspace{6pt}
\textbf{ISM\_configFile:}     [STRING] [optional]                          \\
The name of the ISM-specific config file.                                  \\
\vspace{6pt}
\textbf{FISOC\_ISM\_ReqVars:} [STRING] [required]                          \\
List of variable names required to be provided by the ISM.                 \\
\vspace{6pt}
\textbf{ISM\_ReqVars:}        [STRING] [optional]                          \\
Corresponding exactly to the above, but model-specific in case of different naming conventions. \\
\vspace{6pt}
\textbf{FISOC\_ISM\_DerVars:} [STRING] [required]                          \\
List of variables derived by FISOC from the ISM vars.                      \\ 
To be calculated from ISM vars by hard coded routines in FISOC\_ISM.       \\
\vspace{6pt}

\textbf{OM\_configFile:}      [STRING] [optional]                          \\
The name of the ISM-specific config file.                                  \\
\vspace{6pt}
\textbf{FISOC\_OM\_ReqVars:}  [STRING] [required]                          \\
List of variable names required to be provided by the OM.                  \\
\vspace{6pt}
\textbf{OM\_ReqVars:}         [STRING] [optional]                          \\
Corresponding exactly to the above, but model-specific in case of different naming conventions. \\
\vspace{6pt}
\textbf{OM\_ReqVars\_stagger:} [STRING] [optional]                         \\
Corresponding exactly to the above, descriptions of the grid stagger for each var. \\
\vspace{6pt}
\textbf{FISOC\_OM\_DerVars:}  [STRING] [required]                          \\
List of variables derived by FISOC from the OM vars.  
To be calculated from ISM vars by hard coded routines in FISOC\_OM.        \\
\vspace{6pt}
\textbf{OM\_writeNetcdf:}   [LOGICAL] [optional]                           \\
Switch for dumping the OM import and export variables to NetCDF file.
Defaults to .TRUE.                                                         \\
\vspace{6pt}


\textbf{OM\_outputInterval:} [INTEGER][optional]                           \\
FISOC collects OM output once every OM\_outputInterval OM timesteps. 
Defaults to 1.  dt\_ratio/OM\_outputInterval must be integer.              \\
\vspace{6pt}
\textbf{OM\_dt\_sec:}         [INTEGER][required]                          \\
OM timestep length in seconds.                                             \\
\vspace{6pt}
\textbf{dt\_ratio:}          [INTEGER][required]                           \\
ISM/OM timestep ratio.                                                     \\
\vspace{6pt}
\textbf{start\_year:}        [INTEGER][required]                           \\
Start year and month define the start time of the coupled simulation.      \\
\vspace{6pt}
\textbf{start\_month:}       [INTEGER][required]                           \\
\vspace{6pt}
\textbf{end\_year:}          [INTEGER][required]                           \\
End year and month define the finish time of the coupled simulation.       \\
\vspace{6pt}
\textbf{end\_month:}         [INTEGER][required]                           \\
\vspace{6pt}
\textbf{verbose\_coupling:}  [LOGICAL][required]                           \\
Print FISOC comments to screen (a log file will be written independently of the settings of this value).\\
\end{flushleft}

\subsubsection{FISOC variables}

The union of \textbf{FISOC\_ISM\_ReqVars}, \textbf{FISOC\_ISM\_DerVars}, \textbf{FISOC\_OM\_ReqVars} 
and \textbf{FISOC\_OM\_DerVars} describes the full set of variables required by FISOC for a given simulation. 
Valid values are given in Table~\ref{tab:vars}.
Note that the units given in Table~\ref{tab:vars} are suggested units.  FISOC doesn't care about units, but 
the user must ensure unit consistency.  There may be hard coded unit assumptions in the model specific 
wrappers.

***variable names: sort out dddt and dBdt\_l0.  consistent naming convention needed.

\begin{table}
\begin{center}
\begin{tabular}{ l|l|l }
Variable              & Description                                  & Units \\
\hline
ISM\_temperature\_l0  & Ice temperature at the ice ocean interface.  & K \\
ISM\_temperature\_l1  & Ice temperature in the ISM one level above   & K \\
                      & the ice ocean interface.                     &   \\ 
ISM\_z\_l0            & Height relative to sea level of the ice      & m \\
                      & base.                                        &   \\
ISM\_z\_l1            & Height relative to sea level of the first    &   \\
                      & ISM model level above the ice base.          & m \\
ISM\_z\_l0\_previous  & Height relative to sea level of the ice      &   \\
                      & base one timestep previously.                & m \\
ISM\_dTdz\_l0         & Vertical temperature gradient in the ice     &   \\
                      & at the ice base.                             & K/m \\
ISM\_dddt             & Rate of change of the height of the ice      &   \\
                      & base relative to sea level with respect to   &   \\
                      & time.                                        & m/a \\
ISM\_velocity\_l0     & Ice flow velocity at the ice base            & m/a \\
OM\_dBdt\_l0          & Ice shelf basal melt rate                    & m/a \\
OM\_temperature\_l0   & Ocean temperature at the ice ocean           &   \\
                      & interface  ***or is this the ocean model...  &   \\
                      & guessing the ice temperature at the interface&   \\
\end{tabular}
\end{center}
\caption{FISOC standard variables and typical units.}
\label{tab:vars}
\end{table}

As a naming convention, ``z'' refers to the vertical coordinate, and ``l0'' and ``l1'' refer to the 
model level at the ice-ocean interface (this would typically be the lowest level of the ISM or 
the uppermost level of the OM) and one level above it, respectively.

FISOC outputting occurs on the ocean grid, and consists of dumping both the import and 
export fields to netcdf files. 
\textbf{FISOC\_ISM\_ReqVars} may contain variables that are required only so that they can be written 
out on the ocean grid (typically as a sanity check) rather than actuallly being needed by the OM.


The list of derived variables, \textbf{FISOC\_ISM\_DerVars}, indicates which variables are needed by the 
OM but are not calculated by the ISM or its wrapper. 
The methods for calculating the derived variables are hard coded in FISOC\_ISM.f90. 
Valid values for  \textbf{FISOC\_ISM\_DerVars} include:

ISM\_z\_l0\_previous.  This is the depth of ice base at previous ISM timestep. This is simply stored 
in memory.  No calculation is required, but this variable is needed for the other ``derived'' variables. 

ISM\_dTdz\_l0.  Temperature gradient at ice base.  This is calculated as 
\begin{equation}
ISM\_dTdz\_l0 = \frac{ISM\_temperature\_l1 - ISM\_temperature\_l0}{ISM\_z\_l1 - ISM\_z\_l0}
\end{equation}

ISM\_dddt.  Rate of change of depth of ice base.  This is calculated as 
\begin{equation}
ISM\_dddt = \frac{ISM\_z\_l0 - ISM\_z\_l0\_previous}{ISM\_dt}
\end{equation}


\subsection{Timestepping}
Asynchronous timestepping.
***fill in this section we've implemented both tight coupling (ice and ocean both running 
on the same timescale) and loose coupling (for longer time scales where the ocean is run 
to steady state then the ice sheet continues until significant change has occurred in the cavity).


\subsection{Running FISOC with Elmer/Ice} 
For dynamic linked libraries, shared object files may be needed at run time.  
This can be ensured through use of 
the \$LD\_LIBRARY\_PATH environment variable. 

For example (it is assumed \$ELMER\_HOME was set during Elmer installation):
\begin{lstlisting}
export LD_LIBRARY_PATH="$FISOC_ISM_LIBPATH/:$LD_LIBRARY_PATH"
\end{lstlisting}

***something about parallelism in mesh



\subsection{Running FISOC with ROMS}
\label{sec:runningROMS}

FISOC needs to access the shared library at run time.  One way of ensuring this 
is to add the location of the library to the \$ LD\_LIBRARY\_PATH variable, e.g.:
\begin{lstlisting}
export LD_LIBRARY_PATH="$FISOC_OM_LIBPATH/:$LD_LIBRARY_PATH"
\end{lstlisting}

ROMS writes a lot of information to the screen when run alone.  
When run through FISOC this is redirected to a file named 
\textbf{OM\_stdoutFile}, which must be provided in the FISOC config file 
whenever ROMS is used.

The number of processes to launch FISOC with must be consistent with the number of 
partitions in the ROMS grid.  This is set in the \textbf{OM\_configFile} by the 
Ntile parameters.  For example, the following gives 4 partitions
\begin{lstlisting}
      NtileI == 1                               ! I-direction partition
      NtileJ == 4                               ! J-direction partition
\end{lstlisting}
This could be launched with
\begin{lstlisting}
mpirun -np 4 FISOC_caller
\end{lstlisting}


\subsection{Troubleshooting}

If the error messages to the screen are not helpful, remember to check whether useful 
information has been logged.  By default this will be in files in the current directory 
with names like PETX.FISOC.Log (where X is a process number).

Errors like the following can ocurr (in the log files) when the number of processes is 
not consistent with the number of component partitons (this example involves ROMS):
\begin{lstlisting}
20151119 112603.491 ERROR            PET0 ESMCI_DistGrid.C:1200 ESMCI::DistGrid::create() Value unrecognized or out of range - deBlockList contains out-of-bounds elements
20151119 112603.491 ERROR            PET0 ESMCI_DistGrid_F.C:152 c_esmc_distgridcreatedb() Value unrecognized or out of range Internal subroutine call returned Error
20151119 112603.491 ERROR            PET0 ESMF_DistGrid.F90:1220 ESMF_DistGridCreateDB() Value unrecognized or out of range - Internal subroutine call returned Error
20151119 112603.491 ERROR            PET0 src/FISOC_OM_Wrapper_ROMS.f90:612 Value unrecognized or out of range - Passing error in return code
20151119 112603.491 ERROR            PET0 src/FISOC_OM_Wrapper_ROMS.f90:120 Value unrecognized or out of range - Passing error in return code
\end{lstlisting}

Errors like the following can occur if a component wrapper attempts to access a field 
that has not been created by FISOC, i.e. a field that is not in the list of
required variables in the FISOC config file (Section~\ref{sec:config}) 
(this example involves FISh):
\begin{lstlisting}
20151207 152613.483 ERROR            PET0 ESMCI_Container_F.C:165 ESMCI::Container::get() Invalid argument key does not exist
20151207 152613.484 ERROR            PET0 ESMCI_Container_F.C:448 c_esmc_containergetfield() Invalid argument Internal subroutine call returned Error
20151207 152613.484 ERROR            PET0 ESMF_Container.F90:589 ESMF_ContainerGetField() Invalid argument - Internal subroutine call returned Error
20151207 152613.484 ERROR            PET0 ESMF_FieldBundle.F90:1456 ESMF_FieldBundleGetItem() Invalid argument - Internal subroutine call returned Error
20151207 152613.484 ERROR            PET0 src/FISOC_ISM_Wrapper_FISh.f90:198 Invalid argument - Passing error in return code
\end{lstlisting}



\section{Post processing}

It is expected that the OM and ISM components will output data in their usual formats, 
and that this will form the basis for most output visualisation and post-processing.
FISOC does however have the capcity to output both ISM and OM fields on the OM grid 
in netcdf format.
Scripts for visualising these outputs can be found in the FISOC\_pp directory.
These scripts require a recent python installation and the netcdf4-python module.

***more about these scripts when they actually exist... so far just a brief sanity 
check script exists...




\subsection{Netcdf4-python installation notes}


Depending on your python installation, you may need to run something like this
before installing netcdf4-python:
\begin{lstlisting}
sudo apt-get install python-dev
\end{lstlisting}
More information about the netcdf4-python module can be found here:\\
\url{http://unidata.github.io/netcdf4-python/}\\
\\
You can clone or download the latest netcdf4-python code from the GitHub 
repository:
\begin{lstlisting}
git clone git@github.com:Unidata/netcdf4-python.git 
\end{lstlisting}
Then build and install the netcdf4-python library like this:
\begin{lstlisting}
python setup.py build
python setup.py install
\end{lstlisting}
Install may require root privileges.
You may need to restart your shell.  

A minor discrepancy between the python setup script and the way nc-config 
retrieves relevant flags may lead to errors if mpi wrappers are used. 
This issue should probably be considered a bug in nc-config. 
More information (and a dirty fix) can be found here:\\
\url{https://github.com/Unidata/netcdf4-python/issues/491}\\






\section{Adapting new components to run with  FISOC}
\label{sec:FISOC_SDG}

Any new OM or ISM component to be used with FISOC must first be ESMF compliant.  This basically 
means that it should have an initialise, run and finalise routine. 

FISOC is designed such that the only code developments for new components should be made to the 
model-specific wrappers: FISOC\_OM\_Wrapper\_XXX.f90 and FISOC\_ISM\_Wrapper\_XXX.f90, where 
XXX should be replaced by the component's name.

If it is found that changes to other aspects of the FISOC code are required, this should be 
implemented in collaboration with the FISOC lead developers.

If you wish to submit your new model-specific wrapper to the FISOC repository, please contact 
the FISOC lead developers.


\subsection{Coding practices}

New code should ideally be in Fortran 90 modules.  
All moules should contain the ``implicit none'' statement at the top (immediately after any 
``USE'' statements).  This property will be inherited by all procedures in the module.

Modules should ideally have the private attribute, with only required procedures being 
made public. 

New model-specific wrapper modules should ensure that the initialise, run and finalise 
subroutines are public. 
When writing new wrapper modules, the existing wrapper modules may be used as a template, 
providing examples of the required interfaces.


\subsection{Configuration options}

The configuration file must be called ``FISOC\_config.rc'' and is compatible with ESMF 
config methods.  
An ESMF\_config object is automatically created from this file.

There may be some parameters that are derived from config parameters.  
The FISOC\_utils module provides subroutines under the 
FISOC\_ConfigDerivedAttribute interface 
for obtaining derived config parameters.
These subroutines can be viewed informally as additional methods to complement the ESMF config access methods. 

An advantage of the way the config object is used is that new config arguments can be added and 
used in the coupling without requiring developments to any code other than the new wrapper. 
The config object can be passed to the wrapper and accessed directly. 


\subsection{ISM wrapper}

\subsection{OM wrapper}

\section{Future developments}

\appendix

\section{Pre-requisite installation notes}
\label{app:A}
The following commands worked to install NetCDF and ESMF on a Linux Mint system in 2015.
Some pre-requisites for netcdf were also installed.
The system already had a working OpenMPI installation.



\begin{lstlisting}[language=bash]


# instructions on installing ESMF can be found here:
# http://www.earthsystemmodeling.org/esmf_releases/last_built/ESMF_usrdoc/node9.html

# netcdf instructions
# http://www.unidata.ucar.edu/software/netcdf/docs/netcdf-install.html

cd /somewhere/to/download/and/compile/source/code

sudo apt-get install m4

wget ftp://ftp.unidata.ucar.edu/pub/netcdf/netcdf-4/zlib-1.2.8.tar.gz
tar -xzf zlib-1.2.8.tar.gz 
cd zlib-1.2.8
 ./configure --prefix=/usr/local/
 make check
 sudo -E make install
cd ..

wget ftp://ftp.unidata.ucar.edu/pub/netcdf/netcdf-4/hdf5-1.8.13.tar.gz	
tar -xzf hdf5-1.8.13.tar.gz 
cd hdf5-1.8.13
 # Note the O0 flag in the next line.  The default is O3, 
 # which is strong optimisation.  This can result in failed 
 # checks on some systems.
 CFLAGS="-O0 -fPIC" CC=mpicc CXX=mpiCC FC=mpif90 ./configure --prefix=/usr/local/ --with-zlib=/usr/local  --enable-fortran --enable-parallel --enable-shared
 make check
 sudo -E make install
cd ..

# note that netcdf fortran library is now compiled from a 
# seperate source from the main netcdf c library. Install 
# the c library first, and make sure to create the shared
# object file. 
wget ftp://ftp.unidata.ucar.edu/pub/netcdf/netcdf-4.3.3.tar.gz
tar -xzf netcdf-4.3.3.tar.gz 
cd netcdf-4.3.3/
 LIBS=-ldl CC=mpicc CXX=mpiCC FC=mpif90 CPPFLAGS=-I/usr/local/include/ LDFLAGS=-L/usr/local/lib/  ./configure --prefix=/usr/local --enable-parallel 
 make check
 sudo -E make install
cd ..

wget ftp://ftp.unidata.ucar.edu/pub/netcdf/netcdf-fortran-4.4.2.tar.gz
tar -xzf netcdf-fortran-4.4.2.tar.gz 
cd netcdf-fortran-4.4.2
 LIBS=-ldl CC=mpicc CXX=mpiCC FC=mpif90 LDFLAGS=-L/usr/local/lib/ CPPFLAGS="-I/usr/local/include -DUSE_NETCDF4"  ./configure --prefix=/usr/local
 make check
 sudo -E make install
cd ..

# convenient viewer for contents of netcdf files (not essential)
sudo apt-get install ncview

# ESMF requires ESMF_DIR and probably other environment variables.  
# These can be set at the command line or, for example, in your 
# .bashrc file.  These might work:
export ESMF_DIR="/top/level/directory/for/esmf/"
export ESMF_NETCDF="split"
export ESMF_NETCDF_INCLUDE="/usr/local/include"
export ESMF_NETCDF_LIBPATH="/usr/local/lib"
export ESMF_COMM="openmpi"
export ESMF_PIO="internal"
                                                                                                              
wget downloads.sourceforge.net/project/esmf/ESMF_6_3_0r/ESMF_6_3_0rp1/esmf_6_3_0rp1_src.tar.gz
tar -xf esmf_6_3_0rp1_src.tar.gz
cd esmf 
 make check
 sudo -E make install
cd ..

# In order to actually use ESMF you must set the environment 
# variable ESMFMKFILE.  If you didn't use environment 
# variables to specify the install location this make file 
# will probably end up somewhere like this:
export ESMFMKFILE="$ESMF_DIR/DEFAULTINSTALLDIR/lib/libO/Linux.gfortran.64.openmpi.default/esmf.mk"


\end{lstlisting}

\end{document}
