
\documentclass[12pt]{article}
\begin{document}
\title{ROMS coding for grounding line evolution}
\author{Rupert Gladstone, David Gwyther}
\maketitle{}

Here are some notes relating to recent developments to get MISOMIP simulations running. Some of these notes will find their way in to the FISOC 
GMD paper and or FISOC manual at some point.  They are being checked in here for now just so they don't get lost!


\section{Wetting/drying and ice overburden pressure}

I think Ben tried to tell me at some point that we'd need to do something about pressure, and it may have been in reference to  issues
summarised here, but I don't think I really got what he was talking about at the time.




\subsection{Pressure at the ice-ocean interface}

Test simulations activating wetting/drying in Ben's version of ROMS under grounded ice can give widespread flooding (wetting of dry cells) 
under the grounded ice sheet.
This is due to positive zeta (some kind of dynamic freesurface variable) values increasing the water column thickness above DCRIT (DCRIT being the threshold 
water column thickness for wetting dry cells). 
When the water column becomes active (changes from dry to wet according to DCRIT) for grounded regions the pressure need to increase to represent the 
higher ice overburden pressure, but this was not previously implemented.
The pressure at the ice-ocean interface, $P_{interface}$, was too low because of the assumption of floatation, even under heavily 
grounded ice.  

%GRho=g/rho_0
%GRho0=1000.0*g/rho_0

\begin{eqnarray}
P & = & 1000. \frac{g}{\rho_0}z_w + \frac{g}{\rho_0}(\rho_a+cff2)(z_w-z_r)
\end{eqnarray}


The following equation comes from prsgrd32.h in the ROMS repository and seems to be calculating presure for use in the pressure gradient calculation:
\begin{eqnarray}
\nonumber P & = & \frac{g}{\rho_0}1000.(z_w-z_{ice})- \frac{g}{\rho_0}(\rho_a+0.5\frac{d\rho}{dz}z_{ice})z_{ice}+\frac{g}{\rho_0}(\rho_a+{cff2})(z_w-z_r) \\
            & = & \frac{g}{\rho_0}1000z_w - \frac{g}{\rho_0}(\rho_a+1000+0.5\frac{d\rho}{dz}z_{ice})z_{ice}+\frac{g}{\rho_0}(\rho_a+{cff2})(z_w-z_r) 
\label{eq:prsgrd32} 
\end{eqnarray}
The reason for the slight rearrangement in the second line is because $\rho_a$ in this equation is the ROMS rho, i.e. it is density anomaly rather than 
actual ocean density (took me a while to figure that out).  
The actual ocean water density in the top ROMS layer in this case is $\rho_w = 1000 + \rho_a$.  
So now it can be seen that the middle term in equation~\ref{eq:prsgrd32} gives the pressure at the ice-ocean interface.
%            & = & \frac{g}{\rho_0}\left(1000.(z_w-z_{ice})- (\rho+0.5\frac{d\rho}{dz}z_{ice})z_{ice}+(\rho+{cff2})(z_w-z_r)\right) 
I think the first and last terms relate to extending 
pressure down through the water column.  Interface pressure corresponds to equation 1 in 
``Influence of sea ice cover and icebergs
on circulation and water mass formation in a
numerical circulation model of the Ross Sea,
Antarctica''
By Mike Dinniman et al., except for the division by $\rho_0$. The $P$ in equation~\ref{eq:prsgrd32} above is probably actually 
$P/\rho_0$ due to some internal ROMS things vaguely related to the Boussinesq approximation that I probably do not need to understand 
(note $\rho_0=1025$ in mod\_scalars, though maybe this gets overwritten from the .in file).
So we have 
\begin{eqnarray}
 P_{interface} & = & - {g}(\rho_w+0.5\frac{d\rho}{dz}z_{ice})z_{ice}  \label{eq:DinniPress} \\ 
%\nonumber      & = & - {g \rho z_{ice}} - \frac{g z_{ice}^2}{2} \frac{d\rho}{dz} \\
\nonumber 
\end{eqnarray}

If we want to modify this expression in order to represent grounded ice rather than floating ice then I think 
it should be sufficient to add a term to represent the ice above floatation,  like this:
\begin{eqnarray}
\nonumber P_{interface} & = &  {g \rho_i z_{iaf}} - {g \rho_w z_{ice}} - \frac{g z_{ice}^2}{2} \frac{d\rho}{dz} \\
\nonumber 
\end{eqnarray}
where $\rho_i$ is the density of ice and $z_{iaf}$ is the height above floatation of the ice. 
Note that the first two terms are both positive since $z_{iaf}$ is positive and $z_{ice}$ is 
negative. But ROMS doesn't know enough information to calculate $z_{iaf}$ for grounded ice. 

However, equation~\ref{eq:DinniPress} is an attempt to recreate ice density based on ice draft, ocean water density, 
and the assumption of floatation.
So a neater solution in the long run (especially for fully coupled simulations) might be to 
pass the basal pressure from the ice sheet model and use it to replace this whole expression. 
One problem with this idea is the asynchronicity.  If FISOC passes a rate of change of ice draft to ROMS 
(which it currently does) then ROMS can update 
the ice draft on the ROMS timestep, so from this perspective it is better for ROMS to calculate pressure evolution 
(because if FISOC passes it then it will not update until the next ISM timestep, whereas the geometry is continually evolving). 

An alternative in keeping with the use of DDDT (rate of change of ice draft) to update zice in ROMS is as follows:
\begin{itemize}
\item Add $s_{ice}$ (the height of the upper ice surface) to ROMS in the same derived type (GRID) that contains $z_{ice}$.
\item Add DSDT (rate of change of the upper ice surface) to ROMS in the same derived type (ICESHELFVAR) that contains DDDT.
\item Initialize $s_{ice}$ in the same way as $z_{ice}$ from netcdf (or initialise it through FISOC).
\item Initialize DSDT to inivar (generally zero) in the same way as DDDT (or initialise it through FISOC).
\item Pass DSDT from FISOC to ROMS at run time in the same way as DDDT.
\item Inform ROMS of the ice density in a sensible and consistent (with FISOC and the ISM) way (or, as a temporary 
measure, just hard code the MISOMIP value into the ice shelf mod where it already lives).
\end{itemize}
Now ROMS can calculate the ice overburden pressure directly without having to make assumptions about equivalent water densities: 
\begin{eqnarray}
\nonumber P_{interface} & = & g \rho_i (s_{ice} - z_{ice}) \\
\label{eqn:pinterface}
\end{eqnarray}

So equation~\ref{eq:prsgrd32} would become:
\begin{eqnarray}
\nonumber P & = & \frac{g}{\rho_0}1000 z_w- \frac{g}{\rho_0}\rho_i (s_{ice} - z_{ice})+\frac{g}{\rho_0}(\rho+{cff2})(z_w-z_r) \\
\nonumber 
\end{eqnarray}

One weakness of this approach is that Elmer/Ice is capable of giving the actual normal force at the bed, which 
is not necessarily identical to the ice overburden pressure.  So this approach would be to miss out on 
a potential advantage of Elmer/Ice.  However, Elmer/Ice is too expensive to run on the same timestep as ROMS
so we either have to interpolate Elmer's normal force over time or let ROMS calculate the pressure field. 
The above method is at least consistent with the approach of interpolating ice draft and passing a rate of 
change of ice draft to ROMS.

BTW I noticed there is a variable in GRID called IcePress, but it doesn't seem to get used at all.
We can use this in the future if we decide to use ice pressure (or normal force at the 
boundary) directly from FISOC.





\subsection{Dry to wet pressure gradient}

Another potential problem with directly applying the grounded ice pressure is that it could lead to 
very high pressure gradients (sloping down from inland 
toward the open ocean) near the dry to wet boundary (i.e. grounding line), which may upset ROMS.  
%At the least this would likely cause very strongly negative $\zeta$ under the grounded ice.  
%Do we want to protect this thin layer of water under the grounded ice sheet from high pressures?  
%In fact we see very strongly negative $\zeta$ under the ice shelf near the grounding line, values of 
%-1 to -2m.
We might be able to do this via a limit on height above floatation (say 5m or 10m). 
It is not yet clear whether this will be needed.  
%The latter would be simpler, and perhaps also better, because it is 
%independent of $\zeta$.  What do you think? Do you think something along these lines is necessary, 
%and if so what would be the best method?
%Dave, is it easy to add $s_{ice}$ initialisation in the same way as $z_{ice}$, i.e. 
%reading it in through the netcdf file? 
%Where should I look in the code (and or .in file) to set this up?
%I will access rho\_i from mod\_iceshelf via FISOC and set it to the MISOMIP value.

If we do impose a limit to height above floatation then we would need to calculate it, probably something like this: 
\begin{eqnarray}
\nonumber z_{iaf} & = & total thickness - floating thickness \\
\nonumber         & = & (s_{ice} - z_{ice}) - \frac{\rho_o}{\rho_i} z_{ice} \\
\nonumber         & = & (s_{ice} - z_{ice}) - \frac{(\rho+0.5\frac{d\rho}{dz}z_{ice})}{\rho_i} z_{ice} \\
        & = & s_{ice} - z_{ice} - \frac{\rho}{\rho_i}z_{ice} - \frac{z_{ice}^2}{2 \rho_i}\frac{d\rho}{dz} 
\end{eqnarray}
where $\rho_o$ is the mean ocean water density in a hypothetical water column of depth $z_{ice}$ and 
$\rho$ is the density in ROMS, which in this case is the ocean water density at the ice-ocean interface. 
Then using the above expression for height above floatation the interface pressure would now be
\begin{eqnarray}
\nonumber P_{interface} & = &  {g \rho_i \min({5m,z_{iaf}})} - {g \rho_w z_{ice}} - \frac{g z_{ice}^2}{2} \frac{d\rho}{dz} 
\end{eqnarray}

Hang on, a simpler way:
\begin{eqnarray}
\nonumber z_{fl}  & = &  z_{ice} - \frac{\rho_o}{\rho_i} z_{ice}  \\
\nonumber         & = &  z_{ice} - \frac{\rho}{\rho_i}z_{ice} - \frac{z_{ice}^2}{2 \rho_i}\frac{d\rho}{dz} 
\end{eqnarray}
where $z_{fl}$ is the floatation height above sea level for a given ice draft $z_{ice}$.
This leads to a modified version of equation~\ref{eqn:pinterface}:
\begin{eqnarray}
P_{interface} & = & g \rho_i \left[ \min \left({z_{fl} + 2m,s_{ice}} \right)  - z_{ice} \right]
\end{eqnarray}




\subsection{ru and rv}

Not quite sure what $ru$ and $rv$ are supposed to be, but have something to do with the pressure gradient.
\begin{eqnarray}
\nonumber rv & = & \frac{om\_v}{2}  \Sigma Hz \times \\
\nonumber && \Big[\Delta P - \frac{g}{2 \rho_0}  \Big(\Sigma \rho \Delta z\_r - \frac{1}{5} ( \Delta dRx (\Delta z\_r - \frac{1}{12} \Sigma dZx) \\
          && - \Delta dZx (\Delta \rho - \frac{1}{12}  \Sigma dRx))\Big)\Big]
\end{eqnarray}
Also don't know what $om\_v$, $Hz$, $z\_r$, $dZx$, $dRx$ are. 

In the above, $\Sigma$ means sum over the two neighbouring points in the $j$ direction (presumably used in calculating the mean) 
and $\Delta$ means the difference between neighbouring points in the $j$ direction.  These refer to points $j$ and $j-1$.

Converting the sums to means:
\begin{eqnarray}
\nonumber rv & = & {om\_v} \bar{Hz} \times \\
\nonumber && \Big[\Delta P - \frac{g}{\rho_0}  \Big(\bar{\rho} \Delta z\_r - \frac{1}{10} ( \Delta dRx (\Delta z\_r - \frac{1}{6} \bar{dZx}) \\
          && - \Delta dZx (\Delta \rho - \frac{1}{6} \bar{dRx}))\Big)\Big]
\end{eqnarray}
where the bar means taking the mean across the neighbouring two points in the $j$ direction.
Not knowing what most of these terms mean, we naively impose an upper limit on the pressure 
difference, $\Delta P $.  Note that the pressure difference at the ice-ocean interface is approximated by:
\begin{eqnarray}
\Delta P_{interface}  & = & \rho_i g (H_j - H_{j-1}) \\
               & = & \rho_i g (\Delta H)        \\
\end{eqnarray}
where $H$ is total ice thickness.

If we assume that the high pressure differences are caused by ice thickness gradients rather than other terms, 
we can impose a limit like this: 
\begin{eqnarray}
\Delta P_{max} & = & 2 \rho_i g 
\end{eqnarray}
This is approximately equivalent to limiting the thickness gradient in the ice to 2m per grid cell.

So
\begin{eqnarray}
\nonumber rv & = & {om\_v} \bar{Hz} \times \\
\nonumber && \Big[\min(2 \rho_i g,\Delta P) - \frac{g}{\rho_0}  \Big(\bar{\rho} \Delta z\_r - \frac{1}{10} ( \Delta dRx (\Delta z\_r - \frac{1}{6} \bar{dZx}) \\
          && - \Delta dZx (\Delta \rho - \frac{1}{6} \bar{dRx}))\Big)\Big]
\end{eqnarray}


\subsection{DCRIT and geometric water column thickness}
when DCRIT is set to a higher value than the geometric water column thickness (as defined in the netcdf input file 
by the gap between bathymmetry and ice draft) under grounded ice, positive zeta values are allowed under the grounded 
ice.  Dry cells do, it seems, feel the overburden pressure.  This can result in a kind of halo of positive $\zeta$ 
over several cells near the grounding line.  I decided it is more physically realistic to do as Ben originally 
suggested and set DCRIT to exactly the initial geometric water column thickness under grounded ice.






\section{Updating draft in ROMS}

When we initially activated DDDT (the rate of change of ice draft with time) ROMS 
kept crashing with exit flag 1 (ROMS blows up).  I made some changes, some of which seem to have helped. 



\subsection{Additional checks}

\textbf{Preserving minimum water column thickness}.
Given the bathymmetry does not evolve but ice draft can evolve, the possibility is open for water column 
thickness under grounded ice to be reduced below the initial 20m.  I put in a check and correction for this 
(hard coded in iceshelf\_vbc at time of writing, needs improving!  Maybe put it as a parameter with $\rho_i$). 
I assume Ben had a reason for starting with a minmum 20m gap between bathymmetry and grounded ice everywhere 
in his examples (and Dave has also implemented the same 20m gap in his processing of ocn3 and ocn4 ISOMIP+ 
netcdf files).  So I check for water column thickness below 20m just after updating the iceshelf\_draft 
variable, and raise iceshelf\_draft to preserve 20m water column if needed.  
This is currently not conservative.


\subsection{Ice draft halo}

This is definitely a ROMS itnernal issue and can probably be viewed as a bug fix and 
can now be fogotten about.  
The main cause of the crashes with non-zero DDDT was that the halo of the zice variable was not updated.  
This would not have caused problems for the 
case of ice draft not being updated at run time, and if ice draft was updated in a serial run it would also not 
have been a problem.  I assume this is why Ben did not encounter the problem: I guess he didn't carry out 
parallel runs with an evolving ice draft.

I added an additional call to mp\_exchange2d for zice immediately after updating zice from the iceshelf\_draft 
variable (in the set\_depth code).  I also added a similar call for iceshelf\_draft, though this may be 
superfluous. 



%We could try keeping the original pressure formulation for when the water column thickness drops below some 
%minimum value that we want to maintain, ramping up to the 
%\begin{eqnarray}
%&\nonumber P_{interface}  = - \frac{g \rho z_{ice}}{\rho_0} - \frac{g z_{ice}^2}{2 \rho_0} \frac{d\rho}{dz}                              & \hspace{1cm} (h+\zeta+z_{ice}) < h_{wcm} \\
%&\nonumber P_{interface}  = \left(- \frac{g \rho z_{ice}}{\rho_0} - \frac{g z_{ice}^2}{2 \rho_0} \frac{d\rho}{dz}\right)(1-S) + P_{ISM}S & \hspace{1cm} D_{CRIT} > (h+\zeta+z_{ice}) > h_{wcm} \\
%&\nonumber P_{interface}  = P_{ISM}                                                                                                      & \hspace{1cm} (h+\zeta+z_{ice}) >  D_{CRIT} \\ 
%\nonumber 
%\end{eqnarray}
%where $P_{ISM}$ is pressure at the ice-ocean interface calculated by the ice sheet model, 
%$h_{wcm}$ is the imposed minimum water column thickness (e.g. 20m), 
%and $D_{CRIT}$ is the wetting/drying activation thickness (e.g. 21m).
%$S$ is a scaling factor (between 0 and 1) given by:
%\begin{eqnarray}
%\nonumber S = 
%\frac{(h+\zeta+z_{ice}) - h_{wcm}}{D_{CRIT} - h_{wcm}}  \\
%\nonumber 
%\end{eqnarray}

%Note that I use $h+\zeta+z_{ice}$ as water column thickness.  Zeta ($\zeta$) is included because it can get large under the grounded ice, 
%and we need $\zeta$ as part of the water column calculation in order to trigger the higher (more realistic) pressures.
%so that to this value?  
%I think I probably should add $\zeta$ to this.  If, under the ice, water column thickness (as defined by the difference between
%bedrock and ice lower surface) is 20m and $\zeta$ is 0.5m, what does this mean, physically?  
%The pressure calculation seems to go purely by the geometric water column thickness (i.e. $h+z_{ice}$) but surely we 
%want to impose a higher pressure to stop this subglacial flooding...

\end{document}
